\documentclass{article}
\usepackage[T1,T2A]{fontenc}
\usepackage[utf8]{inputenc}
\usepackage[english,russian]{babel}
\usepackage{xcolor}
\usepackage{amsmath}
\usepackage{amsfonts,bm}
\usepackage{tikz}
\usepackage{listings}
\usetikzlibrary{patterns}
\usetikzlibrary{quotes,angles}
\usepackage{relsize}
\usepackage{float}

\title{Решение Задачи №40}
\author{Серёжи Фуканчика\\34 Ю класс}
\date{Июль 2020}

\begin{document}

\maketitle

\section{Задача}
Решите уравнение:
$$2\sin^2{x}+\sqrt{2}\sin\left(x+\frac{\pi}{4}\right)=\cos{x}$$

\section{Решение}
В этом уравнении сильно мешает отличный от простого $x$ аргумент синуса -- $x+\frac{\pi}{4}$. Попробую избавиться от него с помощью формулы синуса суммы:
$$\sin \left( \alpha \pm \beta \right) = \sin \alpha \cos \beta \pm \cos \alpha \sin \beta $$

получается что:
$$\sin{\left(x+\frac{\pi}{4}\right)}=\sin{x}\cos{\frac{\pi}{4}} + \cos{x}\sin{\frac{\pi}{4}}$$

учитывая, что $\cos{\frac{\pi}{4}}=\sin{\frac{\pi}{4}}=\frac{1}{\sqrt{2}}$:

$$\sin{\left(x+\frac{\pi}{4}\right)}=\frac{\sin{x} + \cos{x}}{\sqrt{2}}$$

Теперь подставлю это в исходное уравнение:

$$2\sin^2{x}+\sqrt{2}\frac{\sin{x} + \cos{x}}{\sqrt{2}}=\cos{x}$$
сокращаю $\sqrt{2}$:
$$2\sin^2{x}+\sin{x} + \cos{x} = \cos{x}$$
косинусы сокращаются:
$$2\sin^2{x}+\sin{x} = 0$$
получается квадратное уравнение. Здесь можно не делать замены переменных, а просто вынести синус за скобку:
$$(\sin{x})(2\sin{x} + 1) = 0$$
и воспользоваться тем, что произведение равно нулю если любой из членов равен нулю. 

Получаю эквивалентную \textit{совокупность} уравнений:

\section{Ответ}

\section{Проверка}

\end{document}

