\documentclass{article}
\usepackage[T1,T2A]{fontenc}
\usepackage[utf8]{inputenc}
\usepackage[english,russian]{babel}
\usepackage{xcolor}
\usepackage{amsmath}
\usepackage{amsfonts}
\usepackage{tikz}

\title{Решение Задачи №8}
\author{Серёжи Фуканчика\\34 Ю класс}
\date{Октябрь 2019}

\begin{document}

\maketitle
\section{Задача}
В зависимости от параметра $a$ решить уравнение:
$$\left[\sqrt{a+\sqrt{a^2-1}}\right]^x+\left[\sqrt{a-\sqrt{a^2-1}}\right]^x=2a$$

\section{Решение}
очевидно, что слагаемые взаимно обратны друг к другу, поэтому поделю и домножу $a+\sqrt{a^2-1}$ на сопряжённое выражение $a-\sqrt{a^2-1}$:
$$\frac{a+\sqrt{a^2-1}}{a-\sqrt{a^2-1}}\cdot{(a-\sqrt{a^2-1})}$$
$$\frac{(a+\sqrt{a^2-1})\cdot{}(a-\sqrt{a^2-1})}{a-\sqrt{a^2-1}}$$
$$\frac{a^2-\sqrt{a^2-1}^2}{a-\sqrt{a^2-1}}$$
квадрат корня это не модуль! модуль это корень квадрата! вот почему модуля нет.
$$\frac{a^2-a^2+1}{a-\sqrt{a^2-1}}$$
$$\frac{1}{a-\sqrt{a^2-1}}$$
подставлю в исходное:
$$\frac{1}{\left[\sqrt{a - \sqrt{a^2-1}}\right]^x}+\left[\sqrt{a-\sqrt{a^2-1}}\right]^x=2a$$
очевидная замена $\left[\sqrt{a-\sqrt{a^2-1}}\right]^x=p$:
$$\frac{1}{p}+p=2a$$
это гипербола
$$\frac{p^2+1}{p}=2\cdot{}a$$
получаются два корня:
$$p_1=a-\sqrt{a^2-1}$$
$$p_2=a+\sqrt{a^2-1}$$
тогда для первого корня
$$\left[\sqrt{a-\sqrt{a^2-1}}\right]^x=a-\sqrt{a^2-1}$$
$$\left[a-\sqrt{a^2-1}\right]^{\frac{x}{2}}=\left[a-\sqrt{a^2-1}\right]^1$$
для второго корня
$$\left[\sqrt{a-\sqrt{a^2-1}}\right]^x=a+\sqrt{a^2-1}$$

\section{Ответ}
$$x=2$$
$$x=-2$$

% https://www.youtube.com/watch?v=KMLvxGsp9C0
\section{Дополнение: сопряжённые выражения}

\end{document}

