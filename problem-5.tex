\documentclass{article}
\usepackage[T1,T2A]{fontenc}
\usepackage[utf8]{inputenc}
\usepackage[english,russian]{babel}
\usepackage{xcolor}
\usepackage{amsmath}
\usepackage{amsfonts}
\usepackage{tikz}

\title{Решение Задачи №5}
\author{Серёжи Фуканчика\\34 Ю класс}
\date{Сентябрь 2019}

\begin{document}

\maketitle

Пределы.
\section{Задча}
$$\lim_{n\to+\infty}{(\sqrt{3n^2+4n+1}-\sqrt{3n^2-2n+5})}$$
проблема тут в том, что у тебя неопределённость $\infty-\infty$. Попробуй превратить её в $\frac{\infty}{\infty}$ которую тебя научили решать в школе.

Для этого нужно домножить и разделить на сопряжённое выражение чтобы избавиться от корня ($(a-b)(a+b)=a^2-b^2$):
$$\frac{(\sqrt{3n^2+4n+1}-\sqrt{3n^2-2n+5})((\sqrt{3n^2+4n+1}+\sqrt{3n^2-2n+5}))}{(\sqrt{3n^2+4n+1}+\sqrt{3n^2-2n+5})}$$
$$\frac{(3n^2+4n+1-3n^2+2n-5)}{(\sqrt{3n^2+4n+1}+\sqrt{3n^2-2n+5})}$$
$$\frac{(6n-4)}{(\sqrt{3n^2+4n+1}+\sqrt{3n^2-2n+5})}$$
теперь подели числитель и знаменатель на $n$:
$$\frac{(6-4/n)}{(\sqrt{\frac{3n^2+4n+1}{n^2}}+\sqrt{\frac{3n^2-2n+5}{n^2}})}$$
$$\frac{(6-4/n)}{(\sqrt{3+4/n+1/{n^2}}+\sqrt{3-2/n+5/{n^2}})}$$
теперь очевидно, когда $n\to\infty$ все части которые делятся на $n$ стремятся к нулю. $4/n\to{}0$, $2/n\to{}0$, $1/{n^2}\to0$, $5/{n^2}\to{}0$.
Остаётся:
$$\frac{6}{\sqrt{3}+\sqrt{3}}=\frac{2\cdot{}3}{2\cdot\sqrt{3}}=\frac{3}{\sqrt{3}}=\sqrt{3}$$

\newpage
$$\sqrt{1+\tg^2(x)}\cdot\frac{\cos^2(-x)\cos(\frac{\pi}{2}-x)}{\tg(\frac{\pi}{2}-x)\sin(\pi-x)}$$

ОДЗ:
$$\sin(\pi-x) \ne 0, x \ne \pi{}n, n\in{}\mathbb{Z}$$ $$\tg(\frac{\pi}{2}-x) \ne 0, x\ne\pi{}k+\frac{\pi}{2}, k\in{}\mathbb{Z}$$
$$x\ne\frac{\pi}{2}+\pi{}m, m\in{}\mathbb{Z}$$
$$x\ne-\pi{}s, s\in{}\mathbb{Z}$$

Зная всё это можно упрощать:
$$\sqrt{1+\tg^2(x)}\cdot\frac{\cos^2(-x){\color{red}\cos(\frac{\pi}{2}-x)}}{\tg(\frac{\pi}{2}-x)\sin(\pi-x)}$$

$${\color{red}\cos{\frac{\pi}{2}-x}}=\sin(x)$$

$$\sqrt{1+\tg^2(x)}\cdot\frac{\cos^2(-x){\color{red}\sin(x)}}{\tg(\frac{\pi}{2}-x)\sin(\pi-x)}$$
дальше
$$\sqrt{1+\tg^2(x)}\cdot\frac{\cos^2(-x)\sin(x)}{\tg(\frac{\pi}{2}-x){\color{red}\sin(\pi-x)}}$$
$${\color{red}\sin(\pi-x)}=\sin(x)$$
$$\sqrt{1+\tg^2(x)}\cdot\frac{\cos^2(-x)\sin(x)}{\tg(\frac{\pi}{2}-x){\color{red}\sin(x)}}$$
сокращу $\sin(x)$:
$$\sqrt{1+\tg^2(x)}\cdot\frac{\cos^2(-x)}{\tg(\frac{\pi}{2}-x)}$$
дальше
$$\sqrt{1+\tg^2(x)}\cdot\frac{\cos^2(-x)}{\color{red}\tg(\frac{\pi}{2}-x)}$$
$${\color{red}\tg(\frac{\pi}{2}-x)}=\ctg(x)=\frac{\cos{x}}{\sin(x)}$$
$$\sqrt{1+\tg^2(x)}\cdot\frac{\cos^2(-x)}{\color{red}\frac{\cos{x}}{\sin(x)}}$$
раз уж косинус - чётная функция получается что $cos^2{(-x)}=cos^2{(x)}$, ну и заодно дробь переверну:
$$\sqrt{1+\tg^2(x)}\cdot\frac{\cos^2(x){\color{red}\sin(x)}}{\color{red}\cos{x}}$$
делю на косинусы:
$$\sqrt{1+\tg^2(x)}\cdot\cos(x)\sin(x)$$
пришла пора посмотреть на $\sqrt{1+\tg^2{x}}$:
$$\sqrt{1+\tg^2(x)}=\sqrt{1+\frac{\sin^2(x)}{\cos^2{x}}}$$
представлю единицу как $\color{red}\frac{\cos^2{x}}{\cos^2{x}}$:
$$\sqrt{{\color{red}\frac{\cos^2{x}}{\cos^2{x}} }+\frac{\sin^2(x)}{\cos^2{x}}}$$
знаменатели одинаковые, так что дроби просто сложить и получается:
$$\sqrt{\frac{\cos^2{x}+\sin^2(x)}{\cos^2{x}}}$$
очевидно,что $\cos^2{x}+\sin^2(x)=1$, так что числитель дроби это просто единица:
$$\frac{1}{\sqrt{\cos^2{x}}}$$
корень из квадрата это модуль, так что всё это упрощается в:
$$\frac{1}{|\cos{x}|}$$
снова запишу полное выражение:
$${\color{red}\sqrt{1+\tg^2(x)} }\cdot\cos(x)\sin(x)$$
т.е.
$${\color{red}\frac{1}{|\cos{x}|} }\cdot\cos(x)\sin(x)$$
получается, что в зависимости от знака косинуса выражение распадается на систему:
\begin{align}
    \begin{cases}
    \frac{\cos(x)\sin(x)}{\cos{x}}\textrm{, если }\cos(x)>0\\
    \frac{\cos(x)\sin(x)}{-\cos{x}}\textrm{, если }\cos(x)<0
    \end{cases}
\end{align}
сокращу косинусы:
\begin{align}
    \begin{cases}
    +\sin(x)&\textrm{, если }\cos(x)>0\\
    -\sin(x)&\textrm{, если }\cos(x)<0
    \end{cases}
\end{align}
решу $\cos{x}>0$:
\begin{align}
    f(x)=\begin{cases}
    +\sin(x) & \textrm{, если }x\in(2\pi{}n-\frac{\pi}{2};2\pi{}n+\frac{\pi}{2})\\
    -\sin(x) & \textrm{, если }x\in(2\pi{}n+\frac{\pi}{2};2\pi{}n+\frac{3\pi}{2})
    \end{cases}
\end{align}
итак, теперь уже можно рисовать график (я его сжал по горизонтали):

\begin{tikzpicture}[xscale=0.5,yscale=1]
\draw[very thin,color=gray] (-8.1,-1.1) grid (8.1,1.1);
\draw[->] (0,-1.2) -- (0,1.2) node[above] {$f(x)$};
\draw[->] (-8.1,0) -- (8.1,0) node[right] {$x$};

\draw[thick,color=blue,samples=50,domain=-7.8:-4.7]   plot (\x,{sin(\x r)});
\draw[thick,color=blue,samples=50,domain=-1.5:1.5]   plot (\x,{sin(\x r)});
\draw[thick,color=blue,samples=50,domain=4.7:7.8]   plot (\x,{sin(\x r)});

\draw[thick,color=red,samples=50,domain=-4.7:-1.5]    plot (\x,{-sin(\x r)});
\draw[thick, color=red,samples=50,domain=1.5:4.7]    plot (\x,{-sin(\x r)});

\end{tikzpicture}

\section{Экспонента}
$$4^{x+1}-33\cdot{}2^{x}+8\le{}0$$
$$4^{x+2}-257\cdot{}2^{x}+16\le{}0$$
$$2^{2x+4}-16\cdot{}2^{x+3}-2^{x+1}+16\le{}0$$
$$2^{x^2}\le{}64$$
$$25^{x}+5^{x+1}+5^{1-x}+\frac{1}{25^x}\le{}12$$
$$2^{x^2}\le{}4\cdot{}2^x$$
$$9^{x+\frac{1}{2}}-28\cdot{}3^{x-1}+1\le0$$
$${\frac{1}{9}}^{9+x}=81^{x}$$
$$7^{4-x}=49$$
$$\log_{7}{49}-4$$
$$\sqrt{5-4\cdot{}x}=5$$
$$\sqrt{3-2\cdot{}x}=-x$$
$$8^{\frac{2x-2}{x}}=\sqrt{4^{x-1}}$$
$$5^x+12^x=13^x$$
$$\sqrt{7+4\sqrt{3}}^{x}+\sqrt{7-4\sqrt{3}}^{x}=14$$
$$2^{x+2}+\sqrt{2^{x+6}\cdot{}5^x}=9\cdot{}5^{x+1}$$
$$\log_{5}{9}\cdot{}\log_{3}{25}$$
$$\log_{2}{25}\cdot{}\log_{5}{4}$$
$$\log_{a}{b^2}\cdot{}\log_{b}{a^2}$$
$$\log_{a}{b}\cdot{}\log_{b}{a}$$
$$\log_{a}{b^n}\cdot{}\log_{b}{a^m}$$
$$(1+\sqrt{5})^{x}+(1+\sqrt{5})^{2x}=(1+\sqrt{5})^{3x}$$
$$(1+\sqrt{5})^x=1/2\cdot(1+\sqrt{5})$$
$$x=1-\log_{1+\sqrt{5}}{2}$$
$$(1+x^2)^2=4x(1-x^2)$$
$$\sqrt{2x+14+8\sqrt{2x-2}}+\sqrt{2x+2-4\sqrt{2x-2}}=6$$
$$x^{\sqrt{x}}=\sqrt{x^x}$$
$${\frac{1}{2}}^{9+x}=16^{x}$$
$$\log_{0.25}{32}$$
$$15\cdot{}3^{\log_{3}{18}}$$
$$\log_{\frac{1}{4}}{32}$$
$$\log_{5}{9}\cdot\log_{3}{25}$$
$$\log_{a}{b^2}\cdot\log_{b}{a^2}$$
$$\log_a{b^n}\cdot\log_b{a^m}$$
$$(1+\sqrt{5})^x+(1+\sqrt{5})^{2x}=(1+\sqrt{5})^{3x}$$
$$\log_{3}{5}+\log_5{27} \vee \frac{\sqrt{50}+\sqrt{24}}{2\sqrt{3}}$$
$$\sqrt{9+4\sqrt{5}}^{\frac{1}{\sin{x}}}+\sqrt{9-4\sqrt{5}}^{\frac{1}{\sin{x}}}=\frac{17}{4}$$
$$\sqrt{2x+14+8\sqrt{2x-2}}+\sqrt{2x+2-4\sqrt{2x-2}}=6$$
$$(\sqrt{x}-1)(\sqrt{2-x}+1)=2(x-1)$$
$$\frac{\sqrt{x-5}}{\log_{\sqrt{2}}{(x-1)}-1}\ge{}0$$

обязательно вставить проверку
сопряжённость
метод переброски
теорема виета

\section{111}

\begin{equation*}
\begin{cases}
x^3-3{}x{}y^2=1 \\
3x^2y-y^3=1 \end{cases}
\end{equation*}

\begin{equation*}
\begin{cases}
(x^2+y^2)xy=30 \\
x^4+y^4=82
\end{cases}
\end{equation*}

minimum:
$$y=\sqrt{x^2-4x+6}$$


$$(1+x^2)^2=4x(1-x^2)$$

$$\left(\sqrt{6\frac{3}{7}}-\sqrt{2\frac{6}{7}}\right):\sqrt{\frac{5}{63}}$$

$$\sqrt[3]{88-50\sqrt{2}}+\sqrt[3]{5\sqrt{2}+7}$$

$$\tg{\frac{\pi{}x}{4}}=-1$$
$$\cos{\frac{\pi{}x}{3}}=\frac{\sqrt{3}}{2}$$
$$-x^2+3x+28\ge{}0$$
$$\cos{20^{\circ}}$$
\

\section{Параметр}
$$x^2+px+3p^4=0$$
$$x^2+5ax+a^4=0$$
$$x^2+px+5p^4=0$$
$$\frac{2p-?+80p^3}{2\sqrt{p^2-20p^4}}$$
$$x^2+3ax+a^4=0$$

\newpage
Вывод формулы перехода к другому основанию: пусть у нас есть логарифм по основанию $b$. нужно перейти к основанию $k$. начнём с основного логарифмического тождества:
$$x=b^{\log_{b}{x}}$$

прологарифмируем обе части по новому основанию:
$$\log_{k}{x}=\log_{k}{b^{\log_{b}{x}}}$$

по формуле степени:
$$\log_{k}{x}=\log_{b}{x}\cdot\log_{k}{b}$$
делим обе части на $\log_k{b}$:
$$\frac{\log_{k}{x}}{\log_k{b}}=\log_{b}{x}$$
т.е.
$$\log_{b}{x}=\frac{\log_{k}{x}}{\log_k{b}}$$

\section{Формулировка}
Серёжа, реши, пожалуйста, уравнение
$$\log_{25}{x}+\log_{5}{x}=\log_{1/5}{\sqrt{8}}$$

\section{Решение}
Что же такое логарифм? К возведению в степень уже есть обратная операция - извлечение корня. Откуда взялся ещё и логарифм?! Оказывается это потому, что операция возведения в степень \textit{некоммутативна}!

Например в сложении мы запросто можем переставить аргументы местами:
$$2+3=3+2 = 5$$
Это позволяет лихо использовать обратную к сложению операцию - вычитание:


\begin{align*} 
2+3=5 | & 5-3=2 \\
3+2=5 | & 5-2=3
\end{align*}

То же самое в умножении:
$$2\cdot{}3=3\cdot{}2=6$$
Тут тоже, не так уже лихо, но всё равно ещё можно (не забывая про ноль):
$$ 6/3=2 $$
$$ 6/2=3 $$

Однако с возведением в степень всё не так:
$$(2^3=8) \ne (3^2=9)$$
$$\sqrt{9}=3, \sqrt[3]{9} \ne 2$$
Оп-па! А что же нам позволит связать $2$, $3$ и $9$ в другую сторону - это логарифм: 

\textbf{Определение:}\textit{Логарифм} - это такая величина степени (назовём $x$), в которую надо возвести \textit{основание} (назовём $a$), чтобы получить число $b$:
$$a^x=b$$
тогда по определению:
$$x=\log_{a}{b}$$

(triangle - POW, LOG, ROOT, RECIPROCAL)
$c=a^b, a=\sqrt[b]{c}, b=log_a{c}$

например:
$$2^3=8$$
$$3=\log_{2}{8}$$
т.е. в какую степень нужно возвести основание $2$, чтобы получить $8$? Ответ, очевидно - три.

Итак, чтобы решить это уравнение, мне понадобятся:
\begin{itemize}
    \item определение логарифма $a^{\log_{a}{b}}=b$;
    \item формула суммы логарифмов $\log_{a}{x\cdot{}y}=\log_{a}{x}+\log_{a}{y}$ (доказать!);
    \item формула логарифма степени $p\cdot\log_{a}{x}=\log_{a}{x^p}$;
    \item формула перехода к другому основанию $\log_{a}{x}=\frac{\log_{b}{x}}{\log_{b}{a}}$.
\end{itemize}

\textbf{Внимание! } В данном случае моя стратегия - используя эти операции и, где можно, обычную алгебру, преобразовать всё это страшное уравнение к простому виду:
$$\log_{a}{x}=\log_{a}{(N)}$$
тогда можно \textit{предположить} что \textit{$x$ равно} $x=N$ где $N$ это какое-то число.

Почему я не написал просто - "тогда $x$ равно $N$"? Также как и с делением на ноль решения может и не существовать. Предположим, что у меня получится что $N$ это ноль или отрицательное число. Тогда, если я решу что $x=0$ или $x\le{}0$, то значит что я решил, что существует какое-то число $b$, такое что (при условии ненулевого $a>0$):
$$a^b=0$$

Может ли быть такое $+2^x<0$?

а этого не может быть. Так что просто приравнять один логарифм к другому не получится. Постоянно нужно следить за областью определения этой функции.

Кстати, хитрые составители задачек могут подкинуть подлянку сделав частный случай с основанием равным нулю или степенью равной нулю, а в этом случае решение может и существовать. Будь внимательна!

Абсолютно аналогично, можно было бы привести вот к какому виду:
$$a^x=a^N$$

\ldots{}опять же $+a^x=-a^x$

откуда тем же способом получается то же самое $a=N$.

Пора вернуться к нашему заданию:
$$\log_{25}{x}+\log_{5}{x}=\log_{1/5}{\sqrt{8}}$$

Главная проблема данного уравнения в том, что логарфим это нелинейная функция ($f(x)+f(y)=f(x+y)$ - аналогия $\frac{a}{b}+\frac{c}{d}$). Это означает, что просто так сложить два логарифма с разными основаниями нельзя. Например $\log_{4}{16}+\log_{2}{8} \ne \log_{4+2}{16+8}$. Более того, даже если у нас есть два логарифма с одинаковыми основаниями складывать их нельзя! 

Контрпример:
$$\log_{2}{8} + \log_{2}{16}$$
тупо складываю \textit{в предположении что логарифм линейная функция}:
$$\log_{2}{(8+16)} = \log_{2}{24}$$
Давай оценим чему равно $\log_{2}{24}$: оно больше $4$ и меньше $5$: $\log_{2}{16}=4 < \log_{2}{24} < \log_{2}{32}=5$. Калькулятор же скажет что оно приблизительно равно $4.58$. Это с одной стороны.

С другой стороны - $\log_{2}(8)=3$ (из определения), $\log_{2}{16}=4$ (из определения). Если просто сложить $3$ и $4$ то получится $7$. Так что $\log_{2}(8)+\log_{2}{16} = 7$. Т.е. $4.58 \ne 7$.

Полученное противоречие указывает, что просто складывать аргументы логарифмов нельзя. И действительно, в школе, среди прочего, проходят формулу суммы логарифмов:
$$\log_{a}{(x\cdot{}y)} = \log_{a}{(x)} + \log_{a}{(y)}$$
т.е. сумма логарифмов равна логарифму произведения!

Давай проверим на одном примере:
$$\log_{2}{8}+\log_{2}{16} = \log_{2}{(8\cdot{}16)} = \log_{2}{128} = 7$$

Ещё прмер $a^{100}+b^{100}$

Напомню - $7$ это степень в которую нужно возвести двойку чтобы получить $128$.

Классно, да? Интуитивно это можно понять вот как. Из правил работы со степенью мы знаем:
$$a^{x}*a^{y} = a^{x+y}$$
т.е. произведение степеней с одним и тем же основанием $a$ степени складываются. А поскольку логарифм это одна из операций обратных к возведению в степень ($a^b=c$, $log_a{c}=b$) то становится понятно и само правило суммы логарифмов. Настоящие доказательства мы рассмотрим позже.

Все логарифмы в этом уравнении имеют основания производные от $5$. Попробую переписать его так, чтобы в основаниях были только пятёрки в разных степенях, а потом применить закон замены основания логарифма (ведь там везде будут целые числа):
$$\log_{5^2}{(x)}+\log_{5^1}{(x)}=\log_{5^{-1}}{\sqrt{8}}$$

По формуле замены основания перейду везде к новому основанию пять (логарифм по новому основанию $c$ равен логарифму по новому делённому на логарифм от старого основания $a$):
$$\log_{a}{b}=\frac{\log_{c}{b}}{\log_{c}{a}}$$
$$\frac{\log_{5}{(x)}}{\log_{5}{5^2}}+\frac{\log_{5}{(x)}}{\log_{5}{5^1}}=\frac{\log_{5}{\sqrt{8}}}{\log_{5}{5^{-1}}}$$
по определению логарифма:
$$\log_{5}{5^2}=2$$
$$\log_{5}{5^1}=1$$
$$\log_{5}{5^{-1}}=-1$$

(как так?! очень просто - $5^x=5^2$)

так что получается:
$$\frac{\log_{5}{(x)}}{2}+\frac{\log_{5}{(x)}}{1}=\frac{\log_{5}{\sqrt{8}}}{-1}$$
поскольку \textit{под логарифм} я не лезу (в данном случае не трогаю $x$ и $\sqrt{8}$), то я могу использовать обычную алгебру и домножить всё на $2$ и чутка подсократить:
$$\log_{5}{(x)}+2\cdot\log_{5}{(x)}=-2\cdot\log_{5}{\sqrt{8}}$$

Теперь я воспользуюсь формулой логарифма степени:
$$\log_{a}{x^p}=p\cdot\log_{a}{x}$$
и внесу двойку и минус двойку под логарифм:
$$\log_{5}{(x)}+\log_{5}{x^2}=\log_{5}{(\sqrt{8)}^{-2}}$$
очевидно (скажи мне если это тебе не очевидно!!!), что
$$\sqrt{8}^{-2} =\frac{1}{\sqrt{8}^2}=\frac{1}{8}$$
тогда наше уравнение:
$$\log_{5}{(x)}+\log_{5}{x^2}=\log_{5}{1/8}$$
напомню, что моя цель - получить
$$\log_{a}{x}=\log_{a}{(N)}$$
так что я опять применю правило сложения логарифмов к левой части и получу:
$$\log_{5}{(x\cdot{}x^2)}=\log_{5}{(1/8)}$$
$$\log_{5}{(x^3)}=\log_{5}{(1/8)}$$
теперь,очевидно,
$$x^3 = 1/8$$
откуда не менее очевидно
$$x = 1/2$$

\section{Ответ}
$$x = 1/2$$

\section{Проверка}
$$\log_{25}{x}+\log_{5}{x}=\log_{1/5}{\sqrt{8}}$$
подставляю $x = \color{red}1/2$:
$$\log_{25}{\color{red}1/2}+\log_{5}{\color{red}1/2}=\log_{1/5}{\sqrt{2^3}}$$
$1/2=\color{blue}2^{-1}$, $\log_{25}{x}=\frac{\log_{5}{x}}{\log_5{25}}={\color{green}1/2}\cdot\log_{5}{x}$, $\sqrt{8}=\color{red}\sqrt{2^3}$:
$${\color{green}1/2}\cdot{}\log_{5}{\color{blue}2^{-1}}+\log_{5}{\color{blue}2^{-1}}=\log_{1/5}{\color{red}\sqrt{2^3}}$$
просто алгебра - вынесу $\color{red}\log_{5}{2^{-1}}$ за скобки:
$${\color{red}\log_{5}{2^{-1}}}\cdot(1/2+1)=\log_{1/5}{\sqrt{2^3}}$$
очевидно, что $1/2+1=\color{red}3/2$
$${\color{red}3/2}\cdot\log_{5}{2^{-1}}=\log_{1/5}{\sqrt{2^3}}$$
тут я занесу $\color{red}3/2$ под логарифм по формуле логарифма степени:
$$\log_{5}{2^{-\color{red}3/2}}=\log_{1/5}{\sqrt{2^3}}$$
я уже переводил $1/5$ в вид $5^{-1}$, так что сейчас, для разнообразия, поступлю наоборот - вынесу минус из степени по формуле степени:
$${\color{red}-}\log_{5}{2^{3/2}}=\log_{1/5}{2^{3/2}}$$
дальше, переход к другой степени учитывая, что $-1=\frac{1}{-1}=\frac{1}{\log_{5}{5^{-1}}}=\color{red}\frac{1}{\log_{5}{1/5}}$:
$$\frac{\log_{5}{2^{3/2}}}{\color{red}\log_{5}{1/5}}=\log_{1/5}{2^{3/2}}$$
и тут я могу перейти к другому основанию ($\color{red}1/5$):
$$\log_{\color{red}1/5}{2^{3/2}}=\log_{1/5}{2^{3/2}}$$
Ништяк! Всё сошлось.

\section{Немного истории}
Лень. Но там классная была история в которой засветились разные крутые математики.

\end{document}

