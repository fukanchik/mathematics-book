\documentclass{article}
\usepackage[T1,T2A]{fontenc}
\usepackage[utf8]{inputenc}
\usepackage[english,russian]{babel}
\usepackage{xcolor}
\usepackage{amsmath}
\usepackage{amsfonts}
\usepackage{tikz}

\title{Решение Задачи №7}
\author{Серёжи Фуканчика\\34 Ю класс}
\date{Октябрь 2019}

\begin{document}

\maketitle
Найди предел:
$$\lim_{n\to+\infty}{\left(\sqrt{3n^2+4n+1}-\sqrt{3n^2-2n+5}\right)}$$
Проблема тут в том, что у тебя неопределённость $\infty-\infty$. Попробуй превратить её в $\frac{\infty}{\infty}$ которую тебя научили решать в школе.

Для этого нужно домножить и разделить на сопряжённое выражение чтобы избавиться от корня $(a-b)(a+b)=a^2-b^2$:
$$\frac{\left(\sqrt{3n^2+4n+1}-\sqrt{3n^2-2n+5}\right)\left(\sqrt{3n^2+4n+1}+\sqrt{3n^2-2n+5}\right)}{\sqrt{3n^2+4n+1}+\sqrt{3n^2-2n+5}}$$
$$\frac{3n^2+4n+1-3n^2+2n-5}{\sqrt{3n^2+4n+1}+\sqrt{3n^2-2n+5}}$$
$$\frac{6n-4}{\sqrt{3n^2+4n+1}+\sqrt{3n^2-2n+5}}$$
теперь подели числитель и знаменатель на $n$:
$$\frac{6-4/n}{\sqrt{\frac{3n^2+4n+1}{n^2}}+\sqrt{\frac{3n^2-2n+5}{n^2}}}$$
$$\frac{6-4/n}{\sqrt{3+4/n+1/{n^2}}+\sqrt{3-2/n+5/{n^2}}}$$
теперь очевидно, когда $n\to\infty$ все части которые делятся на $n$ стремятся к нулю. $4/n\to{}0$, $2/n\to{}0$, $1/{n^2}\to0$, $5/{n^2}\to{}0$.
Остаётся:
$$\frac{6}{\sqrt{3}+\sqrt{3}}=\frac{2\cdot{}3}{2\cdot\sqrt{3}}=\frac{3}{\sqrt{3}}=\sqrt{3}$$

\section{Ответ}
$$\sqrt{3}$$

\end{document}

