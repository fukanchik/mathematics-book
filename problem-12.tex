\documentclass{article}
\usepackage[T1,T2A]{fontenc}
\usepackage[utf8]{inputenc}
\usepackage[english,russian]{babel}
\usepackage{xcolor}
\usepackage{amsmath}
\usepackage{amsfonts}
\usepackage{tikz}
\usetikzlibrary{patterns}
\usepackage{relsize}

\title{Решение Задачи №12}
\author{Серёжи Фуканчика\\34 Ю класс}
\date{Декабрь 2019}

\begin{document}

\maketitle

\section{Задача}
Решить систему
\begin{equation*}
    \begin{cases} 
    6\cdot\lg{\sqrt{x}}+3\cdot{}2^y=5 \\ 10\cdot\lg{x}+3\cdot{}4^y=17
    \end{cases}
\end{equation*}

\section{Решение}
ОДЗ=$x>0$.

$\lg{\sqrt{x}}=\lg{x^{1/2}}=\frac{1}{2}\lg{x}$. Это можно использовать в первом уравнении:
\begin{equation*}
    \begin{cases} 
    3\cdot\lg{x}+3\cdot{}2^y=5 \\ 10\cdot\lg{x}+3\cdot{}4^y=17
    \end{cases}
\end{equation*}
$4^y=2^2\cdot{}y$, а это можно использовать во втором уравнении:
\begin{equation*}
    \begin{cases} 
    3\cdot\lg{x}+3\cdot{}2^y=5 \\ 10\cdot\lg{x}+3\cdot{}2^{2\cdot{}y}=17
    \end{cases}
\end{equation*}
Попробую замену: $t=\lg{x}$, $s=2^y$:
\begin{equation*}
    \begin{cases} 
    3\cdot{}t+3\cdot{}s=5 \\ 10\cdot{}t+3\cdot{}s^2=17
    \end{cases}
\end{equation*}
Из первого выражу $t$ через $s$
$$3\cdot{}t+3\cdot{}s=5$$
$$3\cdot{}t=5-3\cdot{}s$$
$$t=5/3-s$$
и подставлю его во второе:
$$10\cdot{}t+3\cdot{}s^2=17$$
$$10\cdot{}(5/3-s)+3\cdot{}s^2=17$$
$$50/3-10\cdot{}s+3\cdot{}s^2=17$$
$17=51/3$. Использую это:
$$50/3-10\cdot{}s+3\cdot{}s^2=51/3$$
переношу всё влево:
$$3\cdot{}s^2-10\cdot{}s+50/3-51/3=0$$
$$3\cdot{}s^2-10\cdot{}s-1/3=0$$
нахожу корни:
$$s_{1,2}=\frac{10\pm{}\sqrt{100-4\cdot{}3\cdot{}1/3}}{6}=\frac{10\pm{}\sqrt{104}}{6}=\frac{5\pm{}\sqrt{26}}{3}$$

Сравню $5$ и $\sqrt{26}$: $25<26$ следовательно, корень $(5-\sqrt{26})/3$ лишний. Остаётся:
$$s=\frac{5+\sqrt{26}}{3}$$
Обратная подстановка:
$$2^y=\frac{5+\sqrt{26}}{3}$$
откуда
$$y=\log_{2}{\frac{5+\sqrt{26}}{3}}$$
Вспоминаю, что $t=5/3-s$:
$$t=5/3-\frac{5+\sqrt{26}}{3}$$
$$t=\frac{5-5-\sqrt{26}}{3}=-\frac{\sqrt{26}}{3}$$
обратная подстановка:
$$\lg{x}=t=-\frac{\sqrt{26}}{3}$$
$$10^{\lg{x}}=10^{ -\frac{\sqrt{26}}{3} }$$
По определению логарифма:
$$x=10^{-\frac{\sqrt{26}}{3} }$$

\section{Ответ}
$$x=10^{-\frac{\sqrt{26}}{3} }$$
$$y=\log_{2}{\frac{5+\sqrt{26}}{3}}$$

\section{Проверка}
\begin{equation*}
    \begin{cases} 
    6\cdot\lg{\sqrt{x}}+3\cdot{}2^y=5 \\ 10\cdot\lg{x}+3\cdot{}4^y=17
    \end{cases}
\end{equation*}

Первое уравнение:
$$\left.6\cdot\lg{\sqrt{x}}+3\cdot{}2^y=5\right|_{x,y}$$
$$6\cdot\lg{\sqrt{10^{-\frac{\sqrt{26}}{3} }}}+3\cdot{}2^{ \log_{2}{\frac{5+\sqrt{26}}{3}} }=5$$
$$6\cdot{}\frac{1}{2}\cdot\lg{10^{-\frac{\sqrt{26}}{3} }}+3\cdot{}\frac{5+\sqrt{26}}{3}=5$$
$$3\cdot-\frac{\sqrt{26}}{3}+5+\sqrt{26}=5$$
$$-\sqrt{26}+5+\sqrt{26}=5$$
$$5=5$$

Второе уравнение:
$$\left.10\cdot\lg{x}+3\cdot{}4^y=17\right|_{x,y}$$
$$10\cdot\lg{ 10^{-\frac{\sqrt{26}}{3} } }+3\cdot{}4^{ \log_{2}{\frac{5+\sqrt{26}}{3}} }=17$$
$$10\cdot\lg{ 10^{-\frac{\sqrt{26}}{3} } }+3\cdot{}2^{ 2\cdot\log_{2}{\frac{5+\sqrt{26}}{3}} }=17$$
$$10\cdot\lg{ 10^{-\frac{\sqrt{26}}{3} } }+3\cdot{}2^{ \log_{2}{\left(\frac{5+\sqrt{26}}{3}\right)^2} }=17$$
$$10\cdot-\frac{\sqrt{26}}{3} + 3\cdot{}\left(\frac{5+\sqrt{26}}{3}\right)^2 = 17$$
$$-10\cdot\frac{\sqrt{26}}{3} + \frac{(5+\sqrt{26})^2}{3} = 17$$
$$-10\cdot\sqrt{26} + (5+\sqrt{26})^2 = 17\cdot{3}$$
$$25+2\cdot{}5\cdot{}\sqrt{26}+26-10\cdot\sqrt{26} = 17\cdot{3}$$
$$51+10\sqrt{26}-10\sqrt{26} = 51$$
$$51 = 51$$
Всё ок.

\end{document}

