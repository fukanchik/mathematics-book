\documentclass{article}
\usepackage[T1,T2A]{fontenc}
\usepackage[utf8]{inputenc}
\usepackage[english,russian]{babel}
\usepackage{xcolor}
\usepackage{amsmath}
\usepackage{amsfonts,bm}
\usepackage{tikz}
\usetikzlibrary{patterns}
\usetikzlibrary{quotes,angles}
\usepackage{relsize}

\title{Решение Задачи №35}
\author{Серёжи Фуканчика\\34 Ю класс}
\date{Январь 2020}

\begin{document}

\maketitle

\section{Задача}
Решите неравенство:
$$25^x+3\cdot{}10^x-4\cdot{}4^x>0$$

\section{Решение}
Итак, есть:
$$25^x+3\cdot{}10^x-4\cdot{}4^x>0$$
Разложу его вот так:
$$5^x\cdot{}5^{x}+3\cdot{}2^x\cdot{}5^x-4\cdot{}2^x\cdot{}2^x>0$$

рассмотрим основания и только их, это:
$$5*5$$
$$2*5$$
$$2*2$$
они все разные. хорошо бы их было превратить в одно и то же, например попробовать поделить на что-то. Если поделить на 5 то проблемой будет 3 основание 2*2, если поделить на 2 то поблемой будет первое - 5*5. Остаётся только средний вариант - поделить на 2*5. Делю, получаю:
$$5*5/2*5=5/2$$
$$2*5/2*5=1$$
$$2*2/2*5=2/5$$
отлично! у меня есть
\begin{samepage}
$$\frac{5}{2}^x$$
$$1 \textrm{(без икса вообще)}$$
и 
$$\frac{1}{5/2}^x$$
\end{samepage}
вот и всё. Пусть $(5/2)^x=t$, тогда у меня:
$$t+3-\frac{4}{t}>0$$
теперь я сходу знаю что
$$t\ne{}0$$
$t$ - положительное, так что знаменатель на знак нравенства не влияет, поэтому остаётся просто:
$$t^2+3t-4>0$$

который легко решается и даёт два корня
$t_1=-4, t_2=1$ -4 не подходит потому что $(5/2)^x$ никак не может равняться отрицательному числу остаётся только $1$:
$$(5/2)^x=1$$
очевидно что $x=0$ теперь, поскольку $5/2$ \textit{больше единицы} то знак неравенства не менялся так что я имею
$$x>0$$
я очень надеюсь что все моменты тут ясны
например связь ЗАМЕНЫ со ЗНАКОМ неравенства
также, важная часть - "так что знаменатель на знак нравенства не влияет"

схема решения тут чем-то отдалённо напоминает возвратные уравнения которые мы когда-то рассматривали.

Другой вариант деления - разделить на $4^x$. Получим показатели $5^2, 5^1, 1$.

\section{Ответ}
$$x>0$$

\section{Проверка}

\end{document}
% https://math-ege.sdamgia.ru/test?pid=508569

