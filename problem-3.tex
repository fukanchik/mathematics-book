\documentclass{article}
\usepackage[T1,T2A]{fontenc}
\usepackage[utf8]{inputenc}
\usepackage[english,russian]{babel}
\usepackage{amsmath}
\usepackage{autobreak}
\usepackage{xcolor}

\allowdisplaybreaks

\title{Решение Задачи №3}
\author{Серёжи Фуканчика\\33 Ю класс}
\date{Февраль 2019}

\begin{document}

\maketitle
Логарифм.
\section{Задача А78}
Упростите выражение:

$$11^{\log_{3}{\frac{1}{3}}+\log_{11}{7}}$$

\section{Решение}
Здесь в основном основания $11$ за исключением $\log_{3}{\frac{1}{3}}$.

Моя стратегия - превратить эту тройку в $11$, затем привести подобные с помощью формулы суммы логарифмов (там же ведь сумма,да?) и посмотреть что получится.

Чтобы избавиться от тройки можно воспользоваться формулой перехода к другому основанию:

$$\log_{a}{b} = \frac{\log_{c}{b}}{\log_{c}{a}}$$

Но это муторно. Есть более простой путь - очевидно, что $\log_{3}{\frac{1}{3}}$ это $-1$. Это та степень в которую надо возвести $3$ чтобы получилось $\frac{1}{3}$

Перепишу исходное с этим упрощением:

$$11^{-1+\log_{11}{7}}$$

Уже выглядит гораздо лучше!

Теперь воспользуюсь формулой суммы логарифмов одного основания:

$$\log_{a}{x\cdot{}y} = \log_{a}{x} + \log_{a}{y}$$

$-1$ для основания $11$ это $\log_{11}{\frac{1}{11}}$, так что заменяю $-1$:

$$11^{\log_{11}{\frac{1}{11}}+\log_{11}{7}}$$

Применяю формулу суммы логарифмов одного основания:

$$\log_{11}{\frac{1}{11}}+\log_{11}{7}=\log_{11}{\frac{1}{11}}\cdot\frac{7}{1}=\log_{11}\frac{7}{11}$$

Снова переписываю начисто ещё более упрощённое выражение:

$$11^{\log_{11}{\frac{7}{11}}}$$

По определению логарифма это $\frac{7}{11}$.

\section{Ответ}
$$\frac{7}{11}$$

\end{document}

