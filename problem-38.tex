\documentclass{article}
\usepackage[T1,T2A]{fontenc}
\usepackage[utf8]{inputenc}
\usepackage[english,russian]{babel}
\usepackage{xcolor}
\usepackage{amsmath}
\usepackage{amsfonts,bm}
\usepackage{tikz}
\usepackage{listings}
\usetikzlibrary{patterns}
\usetikzlibrary{quotes,angles}
\usepackage{relsize}
\usepackage{float}

\title{Решение Задачи №38}
\author{Серёжи Фуканчика\\34 Ю класс}
\date{Февраль 2020}

\begin{document}

\maketitle

\section{Задача}

У треугольника со сторонами $12$ и $15$ проведены высоты к этим сторонам. Высота, проведённая к первой стороне, равна $10$. Найдите длину высоты, проведённой ко второй стороне.

\section{Решение}
Рисунок:
\begin{center}
\begin{tikzpicture}
\draw (0,0) -- (7.5,0);
\draw (3.8,3.4) -- (7.5,0);
\draw [thick,brown] (3.8,3.4) -- (,0.2);

\draw (0,0) -- (4.47,4);

\draw (4.47,0) -- (4.47,4);
\draw [thick,brown] (4.27,0) -- (4.27,0.2);
\draw [thick,brown] (4.47,0.2) -- (4.27,0.2);

\draw (3,0) node [below] {$a=15$};
\draw (1.5,1.8) node [rotate=40] {$b=12$};
\draw (6,1.8) node [rotate=316] {$c=10$};
\draw (4,1.8) node [rotate=0] {$d=?$};

\end{tikzpicture}
\end{center}

\begin{verbatim}
a:15; /* Дано */
b:12; /* Дано */
c:10; /* Дано */
z:sqrt(a^2-c^2); /* Пифагор */
m:a*(b-z)/c; /* Подобие треугольников */
q:m*z/a; /* Подобие треугольников */
p:c-q; /* Длина как сумма длин */
n:p*z/a; /* Подобие треугольников */
d:expand(m+n); /* Длина как сумма длин */
\end{verbatim}

\section{Ответ}

\section{Проверка}

\end{document}

