\documentclass{article}
\usepackage[T1,T2A]{fontenc}
\usepackage[utf8]{inputenc}
\usepackage[english,russian]{babel}
\usepackage{xcolor}
\usepackage{amsmath}
\usepackage{amsfonts}
\usepackage{tikz}
\usetikzlibrary{patterns}
\usepackage{relsize}

\title{Решение Задачи №11}
\author{Серёжи Фуканчика\\34 Ю класс}
\date{Ноябрь 2019}

\begin{document}

\maketitle

\section{Задача}
Решить уравнение
$$\sqrt{\frac{1+x}{x}}+\frac{1}{x}=5$$

\section{Решение}
В подкоренном выражении разделю на $x$:
$$\frac{1+x}{x}=\frac{1}{x}+1$$
получается:
$$\sqrt{\frac{1}{x}+1}+\frac{1}{x}=5$$
буду стремиться к замене всего корня на новую переменную. Значит для $1/x$ не хватает единицы чтобы получить то же самое что и подкоренное выражение. Прибавлю единицу слева и справа от знака равенства:
$$\sqrt{\frac{1}{x}+1}+\frac{1}{x}+1=5+1$$

Теперь можно делать замену (также известную как подстановка):
$$t=\sqrt{\frac{1}{x}+1}, t\ge{}0$$

Получается вот такое квадратное уравнение:
$$t+t^2=6$$
$$t^2+t-6=0$$
Корни можно найти по теореме Виета или с помощью дискриминанта:

\begin{multline*}
t_{1,2}=\frac{-b\pm{}\sqrt{b^2-4ac}}{2a}=\frac{-1\pm{}\sqrt{1-4\cdot{}1\cdot{}-6}}{2} \\ =\frac{-1\pm\sqrt{1+25}}{2}=\frac{-1\pm5}{2}
\end{multline*}

Тогда
$$t_{1,2}=-3;2$$
Корень $-3$ не удовлетворяет условию $t\ge{}0$, так что остаётся только корень $t=2$. Выполняю обратную замену:
$$\sqrt{\frac{1}{x}+1}=2$$
$$\frac{1}{x}+1=4$$
$$\frac{1}{x}=3$$
$$x=\frac{1}{3}$$

\section{Ответ}
Ответ: $x=\frac{1}{3}$

\section{Проверка}
Подставлю $1/3$ в исходное выражение:
$$\left.\sqrt{\frac{1+x}{x}}+\frac{1}{x}=5\right|_{x=1/3}$$
$$\sqrt{\frac{1+1/3}{1/3}}+\frac{1}{1/3}=5$$
$$\sqrt{\frac{3/3+1/3}{1/3}}+\frac{1}{1/3}=5$$
$$\sqrt{4/3\cdot{}3}+3=5$$
$$\sqrt{4}+3=5$$
$$2+3=5$$
$$5=5$$
Верно!

% https://www.youtube.com/watch?v=4_QstKDSJ-Y

\end{document}

