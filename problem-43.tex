\documentclass{article}
\usepackage[T1,T2A]{fontenc}
\usepackage[utf8]{inputenc}
\usepackage[english,russian]{babel}
\usepackage{tikz}
\usetikzlibrary{calc}
\usetikzlibrary{arrows.meta}
\usetikzlibrary{patterns}
\usetikzlibrary{quotes,angles}
\usepackage{amsmath}

\title{Решение Задачи №43}
\author{Серёжи Фуканчика\\35 Ю класс}
\date{Ноябрь 2021}

\begin{document}

\maketitle

\section{Формулировка}
Дан прямоугольный треугольник. Первый катет равен 5. А проекция другого катета на гипотенузу равна 2,25. Найти Гипотенузу. 

\section{Решение}
Для начала сделаю чертёж.

\begin{tikzpicture}
\draw[red,rotate=50,shift={(3,-0.1)}] (0,0.3) -- (-0.3,0.3);
\draw[red,rotate=50,shift={(3,-0.1)}] (-0.3,0.3) -- (-0.3,0);

\draw (0,0) -- (5,0);
\draw (0,0) -- (0,3.75);
\draw (0,3.75) -- (5,0);
\draw[thick] (0,0) -- (2,2.25);
\draw[red] (0,0.3) -- (0.3,0.3);
\draw[red] (0.3,0.3) -- (0.3,0);

\node[left,below] at (0,0) {$A$};
\node[right] at (5,0) {$C$};
\node[above] at (0,3.75) {$B$};
\node[above] at (2,2.25) {$M$};
\node[below,red] at (2.5,0) {$m=5$};
\node[left,red] at (0,1.875) {$k$};
\node[below,red] at (1.2,1.3) {$h$};
\node[above,red,rotate=-35] at (1,3.1) {$z=2.25$};
\node[above,red,rotate=-35] at (3, 1.6) {$x$};

\node at (2,2.25)[orange,circle,fill,inner sep=1.1pt]{};
\end{tikzpicture}

Прямоугольный треугольник $ABC$ с прямым углом $A$. Гипотенуза обозначена $BC$. Катеты треугольника - первый $AC$ обозначим его длину буквой $m$ и по условию задачи нам дано что $m=5$. Второй катет - $AB$ обозначим его длину буквой $k$.

Проекция катета $AB$  на гипотенузу у меня обозначена $BM$ а его длина дана в условии $z=2.25$. Так как $BM$ это \textit{проекция}, то отрезок $AM$ перпендикулярен гипотенузе. Обозначу длину отрезка $|AM|=h$. 

Часть гипотенузы $BC$ мне дана, из условия я знаю что $|BM|=z=2.25$. Оставшуюся неизвестную часть гипотенузы я назову $x$.

Тогда, получается, мне нужно найти $|BC|=|BM|+|MC|=z+x=2.25+x$.

Замечу, что из-за того что проекция образуется с помощью перпендикуляра, у меня образовалось три прямоугольных треугольника -  исходный $ABC$, и два маленьких = $ABM$ и $AMC$.

Кроме того, у меня три неизвестных величины:  $h$, $k$ и $x$. Мне нужно каким-то образом составить и решить систему из трёх уравнений для этих величин.

Составлю уравнения по Теореме Пифагора для каждого прямоугольного треугольника. Сумма квадратов катетов равна квадрату гипотенузы.

Треугольник $ABC$. Катеты $k$ и $m$, гипотенуза - сумма длин $BM$ и $MC$, $z+x$:
$$m^2+k^2=(z+x)^2$$

Треугольник $ABM$. Катеты $h$, $z$, гипотенуза $k$:
$$h^2+z^2=k^2$$

Треугольник $AMC$. Катеты $h$, $x$, гипотенуза $m$:
$h^2+x^2=m^2$

Получается система уравнений:
$$
\begin{cases}
m^2+k^2=(z+x)^2 \\
{\color{red}h^2+z^2=k^2} \\
{\color{blue}h^2+x^2=m^2}
\end{cases}
$$
$k^2$ я могу взять из второго уравнения и подставить в первое:
$$m^2+k^2=(z+x)^2$$
$$m^2+{\color{red}h^2+z^2}=(z+x)^2$$
то же самое с $m^2$:
$${\color{blue}h^2+x^2}+h^2+z^2=(z+x)^2$$

Немного упрощу:
$$h^2+x^2+h^2+z^2=(z+x)^2$$
$$2\cdot{}h^2+x^2+z^2=(z+x)^2$$
$$2\cdot{}h^2+x^2+z^2=z^2+2xz+x^2$$
Сокращу $x^2$ слева и справа:
$$2\cdot{}h^2+z^2=z^2+2xz$$
сокрашу $h^2$:
$$2\cdot{}h^2=2xz$$
Сокращу двойку:
$$h^2=xz$$

Теперь это можно подставить в третье уравнение:
$$h^2+x^2=m^2$$
$$xz+x^2=m^2$$
вспоминаю, что $z=2.25$, а $m=5$ и подставляю числа:
$$2.25\cdot{}x+x^2=5^2$$
$$2.25\cdot{}x+x^2=25$$
перенесу всё влево:
$$x^2+2.25\cdot{}x-25=0$$
получилось обычное квадратное уравнение.
Решу:
$$x_{1,2}=\frac{-b\pm\sqrt{b^2-4ac}}{2a}$$
$$x_{1,2}=\frac{-2.25\pm\sqrt{2.25^2-4\cdot{}1\cdot{}-25}}{2}$$
$$x_{1,2}=\frac{-2.25\pm\sqrt{2.25^2+100}}{2}$$
корень из $2.25^2+100=105.0625=10.25$:
$$x_{1,2}=\frac{-2.25\pm10.25}{2}$$
значит 
$$x_1=\frac{-2.25+10.25}{2}=\frac{8}{2}=4$$
$$x_2=\frac{-2.25-10.25}{2}=\frac{-12.5}{2}=-6.25$$

По условию задачи второй корень нам не подходит ведь длина не бывает отрицательной, так что берём только первый $x=4$.

Теперь можно найти ответ: $z+x=2.25+4=6.25$

\section{Ответ}
Длина гипотенузы равна $6.25$.

\section{Проверка}
Чтобы проверить можно подставить полученные числа $4$ и $6.25$  в разные уравнения для треугольников.

\end{document}

