\documentclass{article}
\usepackage[T1,T2A]{fontenc}
\usepackage[utf8]{inputenc}
\usepackage[english,russian]{babel}
\usepackage{xcolor}
\usepackage{amsmath}
\usepackage{amsfonts,bm}
\usepackage{tikz}
\usetikzlibrary{patterns}
\usetikzlibrary{quotes,angles}
\usepackage{relsize}

\title{Решение Задачи №24}
\author{Серёжи Фуканчика\\34 Ю класс}
\date{Декабрь 2019}

\begin{document}

\maketitle

\section{Задача}
Решите уравнение:

$$\sin^2{}x + \sin{}x{}\cos{}x = 1$$

\section{Решение}
По основному тригонометрическому тождеству
$$1=\sin^2{x}+\cos^2{x}$$
тогда в исходном уравнении можно заменить $1$:
$$\sin^2{}x + \sin{}x{}\cos{}x = \sin^2{x}+\cos^2{x}$$
$\sin^2{x}$ сокращаются, остаётся:
$$\sin{}x{}\cos{}x = \cos^2{x}$$
перенесу влево
$$\sin{}x{}\cos{}x - \cos^2{x}=0$$
вынесу $\cos{x}$ за скобку
$$\cos{}x(\sin{}x - \cos{x}) = 0$$
теперь я воспользуюсь тем фактом, что произведение двух чисел равно нулю, когда хотя бы одно из них равно нулю. Произведение разбивается на совокупность (\textit{объясни - почему именно совокупность а не систему?})
\begin{equation*}
\left[
\begin{array}{l}
\cos{}x  = 0 \\
\sin{}x - \cos{}x = 0
\end{array}
\right.
\end{equation*}

откуда:
\begin{equation*}
\left[
\begin{array}{l}
\cos{}x  = 0 \\
\sin{}x = \cos{}x
\end{array}
\right.
\end{equation*}

\section{Ответ}

\section{Проверка}

\end{document}

