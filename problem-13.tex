\documentclass{article}
\usepackage[T1,T2A]{fontenc}
\usepackage[utf8]{inputenc}
\usepackage[english,russian]{babel}
\usepackage{xcolor}
\usepackage{amsmath}
\usepackage{amsfonts}
\usepackage{tikz}
\usetikzlibrary{patterns}
\usepackage{relsize}

\title{Решение Задачи №13}
\author{Серёжи Фуканчика\\34 Ю класс}
\date{Декабрь 2019}

\begin{document}

\maketitle

\section{Задача}
Решите уравнение:
$$\frac{x^3}{\sqrt{4-x^2}}+x^2-4=0$$

\section{Решение}
Область определения: $4-x^2>0, 4>x^2, x^2<2^2, -2<x<2$.

Перегруппирую кусок справа:
$$\frac{x^3}{\sqrt{4-x^2}}-(4-x^2)=0$$

Замена: $\sqrt{4-x^2}=a$. Работаю с $a$ как с переменной а $x$ с как с константой:
$$\frac{x^3}{a}-a^2=0$$
$$\frac{x^3-a^3}{a}=0$$
$$x^3-a^3=0$$
$$x^3=a^3$$
$$x=a$$
Обратная замена:
$$x=\sqrt{4-x^2}$$
$$x^2=4-x^2$$
$$2\cdot{}x^2=4$$
$$x^2=2$$
$$x=\pm{}\sqrt{2}$$
$x=-\sqrt{2}$ не подходит, потому что $\sqrt{4-x^2} > 0$.

\section{Ответ}
$$x=\sqrt{2}$$

\section{Проверка}
$$\left.\frac{x^3}{\sqrt{4-x^2}}+x^2-4=0\right|_{x=\sqrt{2}}$$
$$\frac{2\cdot{}\sqrt{2}}{\sqrt{4-\sqrt{2}^2}}+\sqrt{2}^2-4=0$$
$$\frac{2\cdot{}\sqrt{2}}{\sqrt{4-2}}+2-4=0$$
$$\frac{2\cdot{}\sqrt{2}}{\sqrt{2}}+2-4=0$$
$$\frac{2\cdot{}\sqrt{2}}{\sqrt{2}}-2=0$$
$$\frac{2\cdot{}\sqrt{2}}{\sqrt{2}}-2=0$$
$$2-2=0$$
$$0=0$$
Всё ок.

\end{document}

