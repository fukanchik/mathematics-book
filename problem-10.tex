\documentclass{article}
\usepackage[T1,T2A]{fontenc}
\usepackage[utf8]{inputenc}
\usepackage[english,russian]{babel}
\usepackage{xcolor}
\usepackage{amsmath}
\usepackage{amsfonts}
\usepackage{tikz}
\usetikzlibrary{patterns}
\usepackage{relsize}

\title{Решение Задачи №10}
\author{Серёжи Фуканчика\\34 Ю класс}
\date{Ноябрь 2019}

\begin{document}

\maketitle

\section{Задача}
Решить уравнение
$$x^2-4x+6=\sqrt{4-\sin^2{(\pi{}x)}}$$

\section{Решение}
Для начала найду ограничения на область определения - подкоренное выражение всегда больше нуля, так что область определения -- $(-\infty;+\infty)$.

Область значений корня справа находится в интервале $y\in[\sqrt{3};\sqrt{4}]$, т.е. корень синусоиды колеблется в интервале  $y\in[\sqrt{3};2]$. Ведь синус принимает значения в интервале $[-1;1]$, а тогда синус в квадрате в интервале $[0;1]$. Т.е. самое больше куда он поднимается это игрек разное двум.

Теперь нужно обратить внимание на параболу слева. Возможно, я буду возводить выражение в квадрат так что могут появиться лишние корни. Поскольку $\sqrt{\ldots}$ справа всегда больше нуля, то корни всего этого уравнения, которые находятся на тех интервалах, где парабола меньше нуля нужно будет отбросить. Нахожу дискриминант:
$$D=b^2-4{}a{}c=4^2-4\cdot{}1\cdot{}6=16-24=-8$$
Дискриминант меньше нуля. Значит корней у этой параболы нет, тогда, поскольку коэффициент при $x^2$ равен плюс единице, ветви параболы направлены вверх. Получается что вся парабола выше оси абсцисс.

Опа! Правая часть весьма ограничена - она в интервале $y\in[\sqrt{3};2]$, а левая только растёт. Может получиться так, что корней вообще нет, если ордината вершины параболы слева выше чем $2$. Ведь все остальные точки параболы ещё выше. Проверю. Абсцисса вершины параболы будет:
$$x=\frac{-b}{2a}=4/2=2$$

Значение параболы в этой точке:
$$2^2-4\cdot{}2+6=4-8+6=2$$

Итак, вершина параболы находится в точке $(2;2)$. И это единственная точка параболы которая находится так низко. Значит, если решение всего уравнения существует, то оно находится в этой и только в этой точке.

Проиллюстрирую это рисунком:
\begin{center}
\begin{tikzpicture}[scale=1.8, domain=-0.5:4, smooth]
\draw[->](-0.1,0)--(5,0);
\draw[->](0,-0.1)--(0,5);
\draw[dashed,gray](2,0)--(2,2);
\draw(0,-0.1)--(0,0.1) node[below=5]{$0$};
\draw(2,-0.1)--(2,0.1) node[below=5]{$2$};
\draw[domain=0.3:3.7,variable=\x,color=blue] plot ({\x},{\x*\x-4*\x+6}) node[left] {$g(x)=x^2-4x+6$};
\draw[samples=100,color=brown,thick] plot function{sqrt(4-sin(pi*x)*sin(pi*x))} node[below] {$f(x)=\sqrt{4-\sin^2(\pi{}x)}$};
\draw[red] (2,2) circle(0.02) node[above] {$(2;2)$};
\end{tikzpicture}
\end{center}

Чтобы проверить что эта точка является решением подставлю $x=2$ в правую часть:
$$\sqrt{4-\sin^2(\pi{}x)}=\sqrt{4-\sin^2(\pi{}2)}$$

Из единичного круга легко понять, что:
$$\sin(\pi\cdot{}2)=0$$
т.е. правая часть равна:
$$\sqrt{4-\sin^2(\pi\cdot{}2)}=\sqrt{4-0^2}=\sqrt{4}=2$$
Отлично! Получается, что точка $(2;2)$ это решение, а по рассуждениям и рисунку выше ясно, что это единственное решение.

\section{Ответ}
$$x=2$$

\end{document}

