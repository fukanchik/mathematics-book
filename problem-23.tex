\documentclass{article}
\usepackage[T1,T2A]{fontenc}
\usepackage[utf8]{inputenc}
\usepackage[english,russian]{babel}
\usepackage{xcolor}
\usepackage{amsmath}
\usepackage{amsfonts,bm}
\usepackage{tikz}
\usetikzlibrary{patterns}
\usetikzlibrary{quotes,angles}
\usepackage{relsize}

\title{Решение Задачи №23}
\author{Серёжи Фуканчика\\34 Ю класс}
\date{Декабрь 2019}

\begin{document}

\maketitle

\section{Задача}
Решите уравнение:

$$3\sin{}2x+2\cos{}2x=3$$

В ответе укажите число, равное сумме корней уравнения, принадлежащих отрезку $A=[\pi/4;\pi]$, при необходимости округлив это число до двух знаков после запятой.

\section{Решение}

\section{Ответ}

\section{Проверка}

\end{document}

