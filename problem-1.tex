\documentclass{article}
\usepackage[T1,T2A]{fontenc}
\usepackage[utf8]{inputenc}
\usepackage[english,russian]{babel}

\title{Решение Задачи №1}
\author{Серёжи Фуканчика\\33 Ю класс}
\date{Октябрь 2018}

\begin{document}

\maketitle
Замена. Возвратное.
\section{Формулировка}
Серёжа, реши, пожалуйста, уравнение
$$(1+2/x)(1+3/x)(4+x)(6+x)=12$$

\section{Решение}

Да как нехер делать! Видно, что это просто рациональное уравнение второй степени, но оно спряталось за всякими дробями и четвёртыми степенями. Его нужно просто вытащить на свет, а потом тупо решить.

\subsection{Вытаскиваю на свет}
$$(1+2/x)(1+3/x)(4+x)(6+x)=12$$
Первые два члена это дроби вида:
$$1+2/x = (x+2)/x$$
$$1+3/x = (x+3)/x$$
вставлю их обратно в исходное уравнение
$$\frac{(x+2)}{x}\frac{(x+3)}{x}=\frac{(x+2)*(x+3)}{x^2}$$
и ещё не забыть $(x+4)(x+6)$!

Получаю
$$\frac{(x+2)(x+3)(x+4)(x+6)}{x^2}=12$$
вверху $4$ степень внизу $2$ получается $4-2=2$ -- т.е. это действительно квадратное уравнение.
Но если получится корень $x=0$ то нужно будет не забыть его исключить.

Где-то я такое видел! Возможно, тут спряталось \textit{возвратное уравнение}! Других идей нет, так что буду приводить к этому виду.

\begin{samepage}
Раскрою скобки и приведу слагаемые:
$$\frac{(x+2)(x+3)(x+4)(x+6)}{x^2}=12$$
$$\frac{(x^2+5x+6)(x^2+10x+24)}{x^2}=12$$
$$ \frac{x^4+15x^3+80x^2+180x+144}{x^2}=12 $$
поделю левую часть на $x^2$:
$$ x^2+15x+80+180/x+144/x^2=12 $$
\ldots{}перегруппирую:
$$ x^2+144/x^2+15x+180/x+80=12 $$
\end{samepage}
Хмм\ldots{} ${x^2+144/x}$ и ${15x+180/x}$ чем-то похожи! Оба члена выглядят как ${x+N/x}$. Буду добиваться максимального совпадения чтобы сделать им \textit{замену}.

\begin{samepage}
Рассмотрю первый член -- ${x^2+144/x}$. Это почти что ${(x+12/x)^2}$:
$$(x+12/x)^2=x^2+144/x^2+2 \cdot x \cdot 12/x=x^2+144/x^2+24 $$
т.е.
$$(x+12/x)^2=x^2+144/x^2+24$$
$$(x+12/x)^2-24=x^2+144/x^2$$
$$x^2+144/x^2=(x+12/x)^2-24$$
так что $ x^2+144/x^2$ это $(x+12/x)^2$ минус лишние $24$.
\end{samepage}

Удачно что ${144=12^2}$. Огромное спасибо составителю задачника!

Затем рассмотрю второй член -- ${15x+180/x}$. Вынесу $15$ за скобку:
$$15\cdot(x+12/x)$$

Теперь можно вернуться к нашему уравнению и переписать его в виде:
$$ x^2+144/x^2+15x+180/x+80=12 $$
помню, что $x^2+144/x^2$ это $(x+12/x)^2-24$, а $15x+180/x$ это $15\cdot(x+12/x)$

Получается очень удобная для \textit{замены} форма уравнения:
$$ (x^2+144/x^2)+15*(x+12/x)+80=12 $$

Сделаю замену ${x+12/x=Q}$, при этом не забуду что $ x^2+144/x^2$ это $(x+12/x)^2$ минус лишние $24$.

\begin{samepage}
Исходное уравнение схлопывается в обычное квадратное уравнение:
$$Q^2-24+15Q+80=12$$
$$Q^2+15Q+44=0$$

Решаю через дискриминант:
$$D=15^2-4\cdot 1 \cdot 44=49=7^2$$
$$q_{1,2}=\frac{-15\pm \sqrt{D}}{2\cdot 1}=\frac{-15\pm 7}{2}$$
и получаю ответ $q_1=-11$, $q_2=-4$.
\end{samepage}

Теперь вспоминаю, что Q=x+12/x. Получается что у меня два уравнения:

$$ x+12/x=-11 $$
$$ x+12/x= -4 $$

это снова квадратные уравнения по той же причине что и в начале:

$$ x^2+11x+12=0 $$
$$ x^2+ 4x+12=0 $$

Ага! Это не система уравнений, а пара уравнений каждое их которых вносит свои корни в окончательный ответ.

По привычке решаю их дискриминантом. У второго он очевидно меньше нуля:
$${D=16-4\cdot 2\cdot 12=-80}$$
В школьной программе его отбрасываем. Поэтому решаю только первое уравнение.

\subsection{Тупо решаю}
$$ x^2+11x+12=0 $$
$$D=11^2-4\cdot 1 \cdot 12=73$$
$$x_{1,2}=\frac{-11\pm \sqrt{D}}{-2}=\frac{-11\pm \sqrt{73}}{2}$$

\section{Ответ}
$$x_1=-\frac{11}{2}-\frac{\sqrt{73}}{2}$$
$$x_2=\frac{\sqrt{73}}{2}-\frac{11}{2}$$

\section{Проверка}
Подставим ответы в исходное уравнение:
$$(1+2/x)(1+3/x)(4+x)(6+x)=12$$
$$x_1=-\frac{11}{2}-\frac{\sqrt{73}}{2}$$
$$(1+\frac{2\cdot 2}{-11-\sqrt{73}})(1+\frac{3\cdot 2}{-11-\sqrt{73}})(4+\frac{-11-\sqrt{73}}{2})(6+\frac{-11-\sqrt{73}}{2})=12$$
$$(1+\frac{4}{-11-\sqrt{73}})(1+\frac{6}{-11-\sqrt{73}})(\frac{8+-11-\sqrt{73}}{2})(\frac{12-11-\sqrt{73}}{2})=12$$
$$(\frac{-11-\sqrt{73}+4}{-11-\sqrt{73}})(\frac{-11-\sqrt{73}+6}{-11-\sqrt{73}})(\frac{-3-\sqrt{73}}{2})(\frac{1-\sqrt{73}}{2})=12$$
$$(\frac{-7-\sqrt{73}}{-11-\sqrt{73}})(\frac{-5-\sqrt{73}}{-11-\sqrt{73}})(\frac{-3-\sqrt{73}}{2})(\frac{1-\sqrt{73}}{2})=12$$
$$\frac{  (-7-\sqrt{73})(-5-\sqrt{73})(-3-\sqrt{73})({1-\sqrt{73}})}{({-11-\sqrt{73}})\cdot ({-11-\sqrt{73}})\cdot {2}\cdot {2}}=12$$
$$\frac{(35 + 7\sqrt{73}+5\sqrt{73}+73)(-3+3\sqrt{73}-\sqrt{73}+73)}{4\cdot ({-11-\sqrt{73}})\cdot ({-11-\sqrt{73}})}=12$$
$$\frac{(108 + 12\sqrt{73})(2\sqrt{73}+70)}{4\cdot ({121+11\sqrt{73}+11\sqrt{73}+73})}=12$$
$$\frac{(216\sqrt{73}+7560+1752+840\sqrt{73})}{4\cdot ({194+22\sqrt{73}})}=12$$
$$\frac{9312+1056\sqrt{73}}{8\cdot ({97+11\sqrt{73}})}=12$$
$$\frac{8\cdot (1164+132\sqrt{73})}{8\cdot ({97+11\sqrt{73}})}=12$$
$$\frac{1164+132\sqrt{73}}{{97+11\sqrt{73}}}=12$$
$$1164/97=12, 132/11=12$$
$$\frac{12\cdot (97+11\sqrt{73})}{{97+11\sqrt{73}}}=12$$
$$12=12$$
Т.е. $x_1=-\frac{11}{2}-\frac{\sqrt{73}}{2}$ действительно корень. Со вторым корнем - аналогично.

\section{Немного истории}
Теория решения алгебраических уравнений развивалась усилиями Древнего Вавилона, Греков, Индусов, Арабов, Европейцев. История эта очень интересная и поучительная.

Много замечательных людей касались этой задачи - Омар Хайам, Диофант, Архимед, Кардано, Тарталья, Виет, Декарт, Галуа, Гаусс, Эйлер.

\section{Дополнение: возвратные уравнения}
% возвратное https://www.youtube.com/watch?v=EwwcxAyHceU
\end{document}

