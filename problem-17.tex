\documentclass{article}
\usepackage[T1,T2A]{fontenc}
\usepackage[utf8]{inputenc}
\usepackage[english,russian]{babel}
\usepackage{xcolor}
\usepackage{amsmath}
\usepackage{amsfonts,bm}
\usepackage{tikz}
\usetikzlibrary{patterns}
\usetikzlibrary{quotes,angles}
\usepackage{relsize}

\title{Решение Задачи №17}
\author{Серёжи Фуканчика\\34 Ю класс}
\date{Декабрь 2019}

\begin{document}

\maketitle

\section{Задача}
Решить уравнение:

\begin{equation}\label{eqn:1}
    1+\cos(x)=2\sin(x)+\sin(2x)
\end{equation}

\section{Решение}
Для начала нужно сделать так, чтобы под всеми тригонометрическими функциями был одинаковый аргумент (\textit{объясни - почему нельзя просто складывать-умножать синусы разных аргументов?}).

Применю формулу синуса двойного угла:

$$\sin(2x)=2\sin(x)\cos(x)$$

получается:

$$1+\cos(x)=2\sin(x)+2\sin(x)\cos(x)$$

справа можно вынести $2\sin(x)$ за скобку, получится:

$$1+\cos{}x = 2\sin{}x(1+\cos{}x)$$

делить на $1+\cos{}x$ я не буду. Нагляднее будет перенести всё в одну сторону и вынести это выражение за скобку:

$$(1+\cos{}x)-2\sin{}x(1+\cos{}x)=0$$
$$(1+\cos{}x)(1-2\sin{}x)=0$$

теперь я воспользуюсь тем фактом, что произведение двух чисел равно нулю, когда хотя бы одно из них равно нулю. Произведение разбивается на совокупность (\textit{объясни - почему именно совокупность а не систему?}):

\begin{equation*}
\left[
\begin{array}{l}
1+\cos{}x  = 0 \\
1-2\sin{}x = 0
\end{array}
\right.
\end{equation*}

откуда:

\begin{equation*}
\left[
\begin{array}{l}
\cos{}x  = -1 \\
\sin{}x = \frac{1}{2}
\end{array}
\right.
\end{equation*}

С косинусом всё понятно - он выдаёт одно решение за оборот. У синуса будет два пересечения с единичной окружностью, в первом и во втором квадрантах.

Я изобразил корни на единичном круге на рис. \ref{fig:f1}.

\begin{figure}[h!]
\centering
\begin{tikzpicture}[scale=3, cap=round,>=latex]
 \draw[->] (-1.5cm,0cm) -- (1.5cm,0cm) node[right,fill=white] {$x$};
 \draw[->] (0cm,-1.5cm) -- (0cm,1.5cm) node[above,fill=white] {$y$};
 \draw[cyan,thick] (0cm,0cm) circle(1cm);

\draw[color=white!30!gray] (0.9,1) node {I};
\draw[color=white!30!gray] (-0.9,1) node {II};
\draw[color=white!30!gray] (-0.9,-1) node {III};
\draw[color=white!30!gray] (0.9,-1) node {IV};

 \draw[red,dashed, thick] (-1.5,0.5) -- (1.5,0.5);
 \draw[black!60!green,dashed,fill=white,thick] (-1,-1.5) -- (-1,1.5);

 \draw (-0.08,0.62) node[color=red] {$\sin{}\theta=\frac{1}{2}$};
 \draw (-0.65,-0.1) node[color=black!60!green] {$\cos{}\theta=-1$};

        \foreach \x/\c in {30/red, 150/red, 180/black!60!green} {
                % lines from center to point
                \draw[gray] (0cm,0cm) -- (\x:1cm);
                % dots at each point
                \filldraw[\c] (\x:1cm) circle(0.5pt);
        }

        \foreach \x/\xtext/\z in {
            30/\frac{\pi}{6}/1.16cm,
            150/\pi-\frac{\pi}{6}=\frac{5\pi}{6}/1.35cm,
            180/\pi/1.12cm
            }
                \draw (\x:\z) node[fill=white] {$\xtext$};

\node[color=orange] at (0.55, 0.15) {$\bm{\frac{\pi}{6}}$};

\node[color=orange] at (-0.55, 0.15) {$\bm{\frac{\pi}{6}}$};

\draw[color=orange, thick, dashed] (0.5,0) arc (0:30:0.5);

\draw[color=orange,thick, dashed] (-0.5,0) arc (180:150:0.5);

\end{tikzpicture}
\caption{Решения уравнения $1+\cos(\theta)=2\sin(\theta)+\sin(2\theta)$}
\label{fig:f1}
\end{figure}

Для первого квадранта решением синуса будет $\frac{\pi}{6}$, для второго -- 

$$\pi-\frac{\pi}{6}=\frac{5\pi}{6}$$

\begin{samepage}
На рис. \ref{f2} я изобразил графики левой и правой части уравнения (\ref{eqn:1}) отдельно, чтобы корни, которые здесь будут являться точками пересечения (или равенства) отдельных графиков, было видно наглядно.

\begin{figure}[ht!]
\centering
\begin{tikzpicture}[scale=1, cap=round,>=latex,, samples = 100,domain = -5:5]
 \draw[->] (-5cm,0cm) -- (5cm,0cm) node[right,fill=white] {$x$};
 \draw[->] (0cm,-2.5cm) -- (0cm,2.5cm) node[above,fill=white] {$y$};
 
\draw[very thick, blue] plot[id=a] function{1+cos(x)};
\draw[thick,red] plot[id=b] function{2*sin(x)+sin(2*x)};

\foreach \x/\y/\xtxt in {
    -3.6651/0.1339/-\frac{7\pi}{6},
    -3.1415/0/-\pi,
    0.5235/1.866/\frac{\pi}{6},
    2.6179/0.1339/\frac{5\pi}{6},
    3.1415/0/\pi
    }
{
 \draw[blue,fill=blue] (\x,\y) circle(0.05cm);
 \draw[blue] (\x,-0.2) node {$\xtxt$};
 \draw[white!30!gray,dashed] (\x,-2.5) -- (\x,2.5);
}

\draw[color=blue] (4.3,0.9) node[fill=white] {$\tiny{1+\cos(x)}$};
\draw[color=red] (4,-2) node[fill=white] {$\tiny{2\sin(x)+\sin(2x)}$};

\end{tikzpicture}
\caption{Графики $\color{blue}f(x)=1+\cos(x)$ и $\color{red}g(x)=2\sin(x)+\sin(2x)$}
\label{f2}
\end{figure}
\end{samepage}

\newpage
\section{Ответ}

\begin{equation*}
\left[
\begin{array}{l}
x=\pi+2\pi{}n, n \in \mathbb{Z} \\[0.5em]
x=\frac{\pi}{6}+2\pi{}k, k \in \mathbb{Z} \\[0.5em]
x=\frac{5\pi}{6}+2\pi{}m, m \in \mathbb{Z}
\end{array}
\right.
\end{equation*}

\section{Проверка}
Попробую подставить $\pi$, $\pi/6$, $5\pi/6$ в уравнение (\ref{eqn:1}):

\paragraph{$\pi$}: 

$$1+\cos(\pi)=2\sin(\pi)+\sin(2\pi)$$
$$1+-1=2\cdot{}0+0$$
$$0=0$$

\paragraph{$\pi/6$}:
$$1+\cos(\pi/6)=2\sin(\pi/6)+\sin(2\pi/6)$$
$$1+\sqrt{3}/2=2\cdot{}1/2+\sqrt{3}/2$$
$$1+\sqrt{3}/2=1+\sqrt{3}/2$$

\paragraph{$5\pi/6$}:
$$1+\cos(5\pi/6)=2\sin(5\pi/6)+\sin(2\cdot{}5\pi/6)$$
$$1+-\sqrt{3}/2=2\cdot{}1/2+-\sqrt{3}/2$$
$$1-\sqrt{3}/2=1-\sqrt{3}/2$$

Вроде ок.

\end{document}

