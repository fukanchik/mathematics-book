\documentclass{article}
\usepackage[T1,T2A]{fontenc}
\usepackage[utf8]{inputenc}
\usepackage[english,russian]{babel}
\usepackage{xcolor}
\usepackage{amsmath}
\usepackage{amsfonts,bm}
\usepackage{tikz}
\usetikzlibrary{patterns}
\usetikzlibrary{quotes,angles}
\usepackage{relsize}
\usepackage{float}

\title{Решение Задачи №37}
\author{Серёжи Фуканчика\\34 Ю класс}
\date{Январь 2020}

\begin{document}

\maketitle

\section{Задача}
Решите неравенство:
$$\frac{4^x-5\cdot{}2^x+6}{1-3^{x-1}}\leq{}2\cdot{}3^x-5\cdot{}2^x+6$$

\section{Решение}
$$\frac{4^x-5\cdot{}2^x+6}{1-3^{x-1}}\leq{}2\cdot{}3^x-5\cdot{}2^x+6$$
ОДЗ! $x\ne{}1$

Большой ошибкой было бы тут поступить как с пропорцияци и крест-накрест умножить на $1-3^{x-1}$. Надеюсь тебе понятно почему. Поэтому переношу всё влево:
$$\frac{4^x-5\cdot{}2^x+6}{1-3^{x-1}}-(2\cdot{}3^x-5\cdot{}2^x+6)\leq{}0$$

теперь можно домножить на $1-3^{x-1}$:
$$\frac{4^x-5\cdot{}2^x+6-(1-3^{x-1})\cdot{}(2\cdot{}3^x-5\cdot{}2^x+6)}{1-3^{x-1}}\leq{}0$$

\subsection{Числитель}
$$4^x-5\cdot{}2^x+6-(1-3^{x-1})\cdot{}(2\cdot{}3^x-5\cdot{}2^x+6)\leq{}0$$

Раскрываю:
$$4^x-5\cdot{}2^x+6-(1-3^{x-1})\cdot{}(2\cdot{}3^x-5\cdot{}2^x+6)\leq{}0$$
$$4^x-5\cdot{}2^x+6-2\cdot{}3^x+5\cdot{}2^x-6+3^{x-1}\cdot{}2\cdot{}3^x-3^{x-1}\cdot{}5\cdot{}2^x+3^{x-1}\cdot{}6\leq{}0$$

сгруппирую члены с одним основанием:
$$2^{2x}-5\cdot{}2^x+5\cdot{}2^x-2\cdot{}3^x+3^{x}\cdot{}\frac{6}{3}+3^{2x}\cdot{}\frac{2}{3}-3^{x}\cdot{}\frac{5}{3}\cdot{}2^x+6-6\leq{}0$$
делаю вычитания
$$2^{2x}+3^{2x}\cdot{}\frac{2}{3}-3^{x}\cdot{}\frac{5}{3}\cdot{}2^x\leq{}0$$

Хоп-па! Такую ситуацию мы уже видели. Я поделю на $2^x\cdot{}3^x$:
$$\left(\frac{2}{3}\right)^{x}+\left(\frac{3}{2}\right)^{x}\cdot{}\frac{2}{3}-\frac{5}{3}\leq{}0$$

Делаю замену:
$$t=\left(\frac{3}{2}\right)^x$$
получается
$$\frac{1}{t}+t\cdot{}\frac{2}{3}-\frac{5}{3}\leq{}0$$
продолжу
$$\frac{3+2\cdot{}t^2-5\cdot{}t}{3t}\leq{}0$$
по замене $3t$ всегда больше нуля поэтому на неравенство не влияет, поэтому смотрю только на числитель:
$$3+2\cdot{}t^2-5\cdot{}t\leq{}0$$
$$t_1=1, t_2=\frac{3}{2}$$
Итак, искомый интервал:
$$t\in{}[1;\frac{3}{2}]$$
Провожу обратную замену, так как основание было больше единицы знак неравенства везде сохранялся:
$$1\leq{}\left(\frac{3}{2}\right)^x\leq{}\frac{3}{2}$$
для наглядности запишу вот как:
$$\left(\frac{3}{2}\right)^0\leq{}\left(\frac{3}{2}\right)^x\leq{}\left(\frac{3}{2}\right)^1$$
откуда очевидно, что:
$$x\in{[0;1)}$$
\subsection{Знаменатель}
Теперь пришло время вспомнить про знаменатель.
$$1-3^{x-1}<0$$
$$1<^{x-1}$$
$$3^0<^{x-1}$$
$$3^{x-1}>^0$$
$$x-1>0$$
$$x>1$$

\section{Метод интервалов}
Нарисую эти промежутки:
\begin{center}
\begin{tikzpicture}
\draw (0.2,-0.5) node {$0$};
\draw (1,-0.5) node {$1$};
\draw[->] (-1cm,0cm) -- (2cm,0cm) node[right,fill=white] {$t$};

%\draw[pattern=north west lines, pattern color=blue,draw=white](8/3,0) arc (180:0:10 and 1);
\clip (-2,-1) rectangle (4,3);
\draw[draw=blue](0,0) arc (0:30:10 and 1);
\draw[draw=blue](1,0) arc (0:180:0.51 and 0.6);
\draw[draw=blue](1,0) arc (180:150:10 and 1);

%---
\filldraw[white!90!gray] (-1,1) circle(5pt);
\draw[color=red] (-1,1) node {$+$};
\filldraw[white!90!gray] (0.5,1) circle(5pt);
\draw[color=red] (0.5,1) node {$+$};
\filldraw[white!90!gray] (2, 1) circle(5pt);
\draw[color=red] (2, 1) node {$-$};

\filldraw[white!90!gray] (-1,1.5) circle(5pt);
\draw[color=black!50!green] (-1,1.5) node {$+$};
\filldraw[white!90!gray] (0.5,1.5) circle(5pt);
\draw[color=black!50!green] (0.5,1.5) node {$-$};
\filldraw[white!90!gray] (2, 1.5) circle(5pt);
\draw[color=black!50!green] (2, 1.5) node {$+$};
%---

\filldraw[blue] (0:1) circle(1.5pt);
\filldraw[white] (0:1) circle(1pt);
\filldraw[blue] (0:0) circle(1pt);

\end{tikzpicture}
\end{center}

Откуда очевидно, что искомые интервалы:
$$x\in[0;1)\cup{}(1;+\infty)$$

\section{Ответ}
$$x\in[0;1)\cup{}(1;+\infty)$$

\section{Проверка}

\end{document}

