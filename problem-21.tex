\documentclass{article}
\usepackage[T1,T2A]{fontenc}
\usepackage[utf8]{inputenc}
\usepackage[english,russian]{babel}
\usepackage{xcolor}
\usepackage{amsmath}
\usepackage{amsfonts}
\usepackage{tikz}
\usetikzlibrary{patterns}
\usepackage{relsize}

\title{Решение Задачи №21}
\author{Серёжи Фуканчика\\34 Ю класс}
\date{Декабрь 2019}

\begin{document}

\maketitle

\section{Задача}
Плоскость $\pi$ пересекает пирамиду $SABC$ по рёбрам $AS$, $AB$, $SC$, $CB$.

Найдите расстояние от точки $S$ до плоскости $\pi$, если известно, что расстояние от точек $A$, $C$ до плоскости $\pi$ равно $\alpha$, $\gamma$ соответственно, отношение объёма пирамиды $SABC$ к площади сечения плоскостью $\pi$ равно $k$, и $CM:MB=l_1:l_2$, где $M$ - точка пересечения плоскости $\pi$ с ребром $BC$ пирамиды.

$$l_1:l_2=1:3, \alpha=5, \gamma=4, k=221/28$$

Если условие задачи допускает несколько геометрических конфигураций, в ответ впишите наибольшее из возможных расстояний. Если описанная в условии конфигурация невозможна, в ответ впишите $0$. При необходимости округлите ответ до одного знака после запятой.

\section{Решение}

\section{Ответ}

\section{Проверка}

\end{document}

