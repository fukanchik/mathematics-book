\documentclass{article}
\usepackage[T1,T2A]{fontenc}
\usepackage[utf8]{inputenc}
\usepackage[english,russian]{babel}
\usepackage{amsmath}
\usepackage{autobreak}
\usepackage{xcolor}

\allowdisplaybreaks

\title{Решение Задачи №2}
\author{Серёжи Фуканчика\\33 Ю класс}
\date{Декабрь 2018}

\begin{document}

\maketitle
Индукция.
\section{Формулировка}
Серёжа, вычисли
$$\sqrt{2019{}\sqrt{2018\sqrt{2017\sqrt{2016\dots\sqrt{3+1}\dots+1}+1}+1}+1}$$

\section{Решение}
Долго пялюсь на это страшное выражение и начинаю подозревать, что внутренние части на разных уровнях вложенности похожи друг на друга.

Проверю явно несколько простейших случаев:
$$\sqrt{3+1}=2$$
$$\sqrt{4{}\sqrt{3+1}+1}=3$$
$$\sqrt{5{}\sqrt{4{}\sqrt{3+1}+1}+1}=4$$

Что-то начинает вырисовываться. Я предполагаю, что:
\begin{equation}\label{eq:1}
\sqrt{n{}\sqrt{\text{\ldots{}всякая фигня\ldots{}}}+1}=n-1
\end{equation}

Доказательство по индукции того что следующая формула верна при $n>=3$:

$$\sqrt{n{}\sqrt{n-1{}\sqrt{n-2{}\sqrt{n-3{}\dots\sqrt{3+1}\dots+1}+1}+1}+1}=n-1$$

\subsection{Базовый случай ($n=3$)}

При $n=3$:
$$\sqrt{3+1}=2$$
проверяется непосредственно.

\subsection{Шаг индукции ($n$ - любое)}

Предположим, что для произвольного $k$:
$$\sqrt{k{}\sqrt{k-1{}\sqrt{k-2{}\sqrt{k-3{}\dots\sqrt{3+1}\dots+1}+1}+1}+1}=\begingroup\color[rgb]{0.1,0.5,0.2}\boldsymbol{k-1}\endgroup$$
при любом $k>=3$, тогда посмотрим на следующий шаг - $k+1$:
\begin{align}
\begin{autobreak}
\MoveEqLeft
\sqrt{\begingroup\color{red}{k+1}\endgroup{}\sqrt{k{}\sqrt{k-1{}\sqrt{k-2{}\sqrt{k-3{}\dots\sqrt{3+1}\dots+1}+1}+1}+1}}=
\text{\textit{(заменим внутренний корень на то что ранее предположили)}}
=\sqrt{\begingroup\color{red}{(k+1)}\endgroup*\begingroup\color[rgb]{0.1,0.5,0.2}\boldsymbol{(k-1)}\endgroup+1}=\sqrt{(k^2-1)+1}=\sqrt(k^2)
=k=(k+1)-1
\end{autobreak}
\end{align}

Получается, что 
$$\sqrt{\boldsymbol{(k+1)}{}\sqrt{\text{\ldots{}всякая фигня\ldots{}}}+1}=\boldsymbol{(k+1)}-1$$

А это именно то, что я предположил вначале в формуле (\ref{eq:1}).

\section{Ответ}
$$\sqrt{2019{}\sqrt{2018\sqrt{2017\sqrt{2016\dots\sqrt{3+1}\dots+1}+1}+1}+1}=2018$$

\section{Немного истории}
Евклид и Блез Паскаль развивали этот метод. Современное название этому методу дал тот самый Де Морган законы которого проходят в школе по информатике.

``Доказательство'' того, что этот метод не просто магия, а действительно работает -- основано на определении натуральных чисел, которое дал математик по имени Джузеппе Пеано.

Метод матиндукции широко используется в современной математике и физике, и с его помощью доказана куча всяких полезностей в матанализе, например, и в квантовой физике.

\end{document}

