\documentclass{article}
\usepackage[T1,T2A]{fontenc}
\usepackage[utf8]{inputenc}
\usepackage[english,russian]{babel}
\usepackage{xcolor}
\usepackage{amsmath}
\usepackage{amsfonts,bm}
\usepackage{tikz}
\usetikzlibrary{patterns}
\usetikzlibrary{quotes,angles}
\usepackage{relsize}

\title{Решение Задачи №27}
\author{Серёжи Фуканчика\\34 Ю класс}
\date{Декабрь 2019}

\begin{document}

\maketitle

\section{Задача}
Турист преодолел маршрут, состоящий из трех участков $AB$, $BC$ и $CD$ равной длины, со средней скоростью $a=3$ км/ч. Его средняя скорость на $AB$ в $k=9/8$ раз больше его средней скорости на пути от $B$ к $D$ и равна полусумме его средних скоростей на $BC$ и $CD$. Определите сумму средних скоростей туриста на участках $AB$ и $CD$, если на прохождение $BC$ он потратил меньше времени, чем на прохождение $CD$.
При необходимости округлите ответ до двух знаков после запятой.

\section{Решение}

\section{Ответ}

\section{Проверка}

\end{document}

