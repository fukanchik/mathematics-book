\documentclass{article}
\usepackage[T1,T2A]{fontenc}
\usepackage[utf8]{inputenc}
\usepackage[english,russian]{babel}
\usepackage{xcolor}
\usepackage{amsmath}
\usepackage{amsfonts}
\usepackage{tikz}
\usetikzlibrary{patterns}
\usepackage{relsize}

\title{Решение Задачи №19}
\author{Серёжи Фуканчика\\34 Ю класс}
\date{Декабрь 2019}

\begin{document}

\maketitle

\section{Задача}
Петя и Аня за урок математики должны решить одинаковое число задач. Через некоторое время после начала урока оказалось, что Петя решил половину того, что осталось решить Ане, а Ане осталось решить треть того, что она уже решила. Аня подсчитала, что если будет продолжать решать задачи с той же скоростью, то она успеет решить все задачи точно к концу урока.

Во сколько раз Пете нужно увеличить свою скорость решения задач, чтобы решить их все к концу урока? Ответ запишите в виде числа.

\section{Решение}

\section{Ответ}

\section{Проверка}

\end{document}

