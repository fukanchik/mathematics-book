\documentclass{article}
\usepackage[T1,T2A]{fontenc}
\usepackage[utf8]{inputenc}
\usepackage[english,russian]{babel}
\usepackage{xcolor}
\usepackage{amsmath}
\usepackage{amsfonts}
\usepackage{tikz}
\usetikzlibrary{patterns}
\usepackage{relsize}

\title{Решение Задачи №9}
\author{Серёжи Фуканчика\\34 Ю класс}
\date{Ноябрь 2019}

\begin{document}

\maketitle

\section{Задача}
Серёжа, реши, пожалуйста, неравенство:
$$\left(\sqrt{3}-\sqrt{2}\right)^{(\log_{2}{3})^{4-x^2}} \leq \left(\sqrt{3}+\sqrt{2}\right)^{-(\log_{3}{2})^{2x-1}}$$

\section{Решение}
сопряжённое:
$$\frac{(\sqrt{3}-\sqrt{2})\cdot(\sqrt{3}+\sqrt{2})}{\sqrt{3}+\sqrt{2}}=\frac{3-2}{\sqrt{3}+\sqrt{2}}=\frac{1}{\sqrt{3}+\sqrt{2}}$$
получается:
$$\left(\frac{1}{\sqrt{3}+\sqrt{2}}\right)^{(\log_{2}{3})^{4-x^2}} \leq \left(\sqrt{3}+\sqrt{2}\right)^{-(\log_{3}{2})^{2x-1}}$$
очевидная замена:
$$t=\sqrt{3}+\sqrt{2}$$
$$\left(\frac{1}{t}\right)^{(\log_{2}{3})^{4-x^2}} \leq \left(t\right)^{-(\log_{3}{2})^{2x-1}}$$
отрицательная степень справа это обратная величина (в чём разница между обратным и противоположным числами?):
$$\left(\frac{1}{t}\right)^{(\log_{2}{3})^{4-x^2}} \leq \left(\frac{1}{t}\right)^{(\log_{3}{2})^{2x-1}}$$
$t>1$, значит $\frac{1}{t}<1$ поэтому чтобы избавиться от показательной функции придётся перебросить знак неравенства:
$${(\log_{2}{3})^{4-x^2}} \geq {(\log_{3}{2})^{2x-1}}$$
переход к другому основанию тут приведёт к обратной величине:
$${(\log_{2}{3})^{4-x^2}} \geq {(\frac{1}{\log_{2}{3}})^{2x-1}}$$
опять замена:
$$k=\log_{2}{3}$$
$${k^{4-x^2}} \geq {\left(\frac{1}{k}\right)^{2x-1}}$$
нужно перейти к одному основанию. Опять воспользуюсь тем что $a^-1=1/a$:
$${\left(\frac{1}{k}\right)^{x^2-4}} \geq {\left(\frac{1}{k}\right)^{2x-1}}$$
$1/k$ очевидно (очевидно ли тебе это девица?) меньше единицы, тогда чтобы избавиться от степени нужно опять перебросить знак неравенства:
$$x^2-4 \leq {2x-1}$$
решаю квадратное неравенство:
$$x^2-4 -2x+1 \leq 0$$
$$x^2-2x-3 \leq 0$$
$$x_{1,2}=\frac{-b\pm{}\sqrt{b^2-4\cdot{}a\cdot{}c}}{2\cdot{}a}=\frac{2\pm{}\sqrt{4+12}}{2}=\frac{2\pm{}\sqrt{16}}{2}=1\pm{}2$$
$$x_1=-1,x_2=3$$
методом интервалов:

\begin{center}
\begin{tikzpicture}[mydrawstyle/.style={draw=black, very thick}, x=1mm, y=1mm, z=1mm]
\draw[pattern=north west lines, pattern color=blue](30,0) arc (0:180:20 and 7);
\draw[draw=blue](-10,0) arc (0:60:20 and 7);
\draw[draw=blue](30,0) arc (180:90:20 and 7);
\draw[mydrawstyle, ->](-20,0)--(50,0);
\draw[mydrawstyle,color=gray](0,-1)--(0,1) node[below=1]{$0$};
\draw[mydrawstyle](-10,-1)--(-10,1) node[below=5]{$-1$};
\draw[mydrawstyle](30,-1)--(30,1) node[below=5]{$3$};
\draw[mydrawstyle](-17,1)node[above=1]{$\color{red}\mathlarger{+}$};
\draw[mydrawstyle](10,1) node[above=1]{$\color{red}\mathlarger{-}$};
\draw[mydrawstyle](40,1) node[above=1]{$\color{red}\mathlarger{+}$};
\end{tikzpicture}
\end{center}

\section{Ответ}
Ответ:
$$x\in{}[-1;3]$$

\section{Проверка}
На всякий случай вставлю ключевые значения $-2$, $-1$, $0$, $3$, $5$.

\section{Дополнение: почему $1/k<1$?}
Напомню -- $k=\log_{2}{3}$. Т.е. по определению логарифма это $2^k=3$. Это число где-то между 1 и 2:
$$2^1=2, 2^k=3, 2^2=4$$
т.е. $1<k<2$, тогда
$$\frac{1}{1}>k>\frac{1}{2}$$
откуда получается, что $1/k<1$.

\section{Дополнение: сопряжённые выражения}
Часто, составители задач подбирают числа таким образом чтобы:
$$a-b=\frac{1}{a+b}$$
Например: $\sqrt{3}-\sqrt{2}=\frac{1}{\sqrt{3}+\sqrt{2}}$.


%Также посмотреть вот эту задачу: https://www.youtube.com/watch?v=X6C5hGpWW5A


%https://www.youtube.com/watch?v=r5mf4UQwEpU

\end{document}

