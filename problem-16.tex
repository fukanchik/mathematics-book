\documentclass{article}
\usepackage[T1,T2A]{fontenc}
\usepackage[utf8]{inputenc}
\usepackage[english,russian]{babel}
\usepackage{xcolor}
\usepackage{amsmath}
\usepackage{amsfonts}
\usepackage{tikz}

\title{Решение Задачи №16}
\author{Серёжи Фуканчика\\34 Ю класс}
\date{Октябрь 2019}

\begin{document}

\maketitle

Каковы должны быть размеры консервной банки цилиндрической формы, чтобы на её изготовление пошло наименьшее количество материала, если объем банки 0,5 литра?

Один литр равен одному кубическому дециметру или тысяче кубических сантиметров.

Решение: в задаче нужно минимизировать функцию $S(r,h)$. При этом $r$ и $h$ взаимосвязаны через указанный фиксированный объём. Поэтому можно выбрать одну переменную, например $r$, а вторую выразить через неё: $h=f(r)$. Получится, что площадь зависит только от одной переменной - $r$: минимизируем $S(r)$.

Общая площадь это площадь стенки плюс дважды площадь дна:
$$S=2\cdot\pi{}r^2+2\pi{}r{}h$$

Объём тоже понятен:
$$V=\pi\cdot{}r^2\cdot{}h$$

Из объёма можно найти $h$:
$$h=\frac{V}{\pi\cdot{}r^2}$$

Получается формула площади зависящая только от радиуса:
$$S(r)=2\cdot\pi{}r^2+2\pi{}r{}\frac{V}{\pi\cdot{}r^2}$$

Подсокращу:
$$S(r)=2\cdot\pi{}r^2+2\frac{V}{r}$$

$V=500\textrm{ см}^3$, так что:
$$S(r)=2\cdot\pi{}r^2+\frac{1000}{r}$$

Поищу минимум или максимум - возьму производную:
$$S'(r)=4\cdot\pi{}r-\frac{1000}{r^2}$$
\ldots{}и приравняю к нулю:
$$4\cdot\pi{}r-\frac{1000}{r^2}=0$$
$$\frac{4\cdot\pi{}r^3-1000}{r^2}=0$$
$$4\cdot\pi{}r^3-1000=0$$
$$4\cdot\pi{}r^3=1000$$
$$r^3=\frac{250}{\pi}$$
$$r=\sqrt[3]{\frac{250}{\pi}}=5\cdot{}\sqrt[3]{\frac{2}{\pi}}$$

Тут придётся взять вторую производную чтобы убедиться что это минимум а не максимум:
$$S''(r)=4\cdot\pi+\frac{1000\cdot{}2}{r^3}$$
подставлю точку:
$$S''(5\cdot{}\sqrt[3]{\frac{2}{\pi}})=4\cdot\pi+\frac{2000}{250/\pi}=12\cdot\pi>0$$

Значит минимум это точка
$$r=5\cdot{}\sqrt[3]{\frac{2}{\pi}}$$
Тогда высота будет:
$$h=\frac{500}{\pi\cdot{}\left(5\cdot{}\sqrt[3]{\frac{2}{\pi}}\right)^2}=\frac{20}{\sqrt[3]{4\cdot\pi}}$$

Ответ: радиус равен
$$r=5\cdot{}\sqrt[3]{\frac{2}{\pi}}\approx{}4.3\textrm{ см}$$

высота равна
$$h=\frac{20}{\sqrt[3]{4\cdot\pi}}\approx{}8.6\textrm{ см}$$

Общая площадь будет:

Проверка: 
$$500=\pi{}r^2\cdot{}h=\pi\cdot{}\left(5\cdot{}\sqrt[3]{\frac{2}{\pi}}\right)^2\cdot{}\frac{20}{\sqrt[3]{4\cdot\pi}}$$
$$\pi\cdot{}500\sqrt[3]{\frac{4}{\pi^2}}\cdot{}\frac{1}{\sqrt[3]{4\cdot\pi}}$$
$$\pi\cdot{}500\frac{1}{\sqrt[3]{\pi^2}}\cdot{}\frac{1}{\sqrt[3]{\pi}}=\pi\cdot{}500\frac{1}{\sqrt[3]{\pi^3}}=\pi\cdot{}500\cdot\frac{1}{\pi}=500$$
Верно

%http://www.mathprofi.ru/zadachi_na_minimumy_i_maksimumy.html

% https://www.coursera.org/learn/calculus1/lecture/W2w5J/how-do-you-design-the-best-soup-can

\end{document}

