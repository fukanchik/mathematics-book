\documentclass{article}
\usepackage[T1,T2A]{fontenc}
\usepackage[utf8]{inputenc}
\usepackage[english,russian]{babel}
\usepackage{xcolor}
\usepackage{amsmath}
\usepackage{amsfonts}
\usepackage{tikz}
\usetikzlibrary{patterns}
\usepackage{relsize}

\title{Решение Задачи №14}
\author{Серёжи Фуканчика\\34 Ю класс}
\date{Декабрь 2019}

\begin{document}

\maketitle

\section{Задача}
Для всех действительных чисел $x$ и $y$ функция $f$ удовлетворяет условию:

$$f(xy-2y)-f(2y^2-x)=f(y)+2x^2y^2-8y^4-2x^2+6y^2+4.$$

Найдите $f(14)$, при необходимости округлив это число до ближайшего целого.

\section{Решение}
Моя тактика самая простейшая - подставлять маленькие значения, такие как -1, 0, 1, и смотреть что получится.

У функции нет ограничений на область определения -- она определена для всех действительных чисел. Я попробую выбирать такие $x$ и $y$, чтобы аргумент получался маленьким или просто $14$, при этом я буду стараться избегать больших $y$ потому что в формуле справа есть $y^4$ что будет неудобно вычислять. При этом для $x$ у меня больше свободы.

\subsection{Случай $x=0$, $y=0$}
Итак, для начала попробую самое простое -- {\boldmath $x=0$, $y=0$}:

$$f(xy-2y)-f(2y^2-x)=f(y)+2x^2y^2-8y^4-2x^2+6y^2+4$$
$$f(0\cdot{}0-2\cdot{}0)-f(2\cdot{}0^2-0)=f(0)+2\cdot{}0^2\cdot{}0^2-8\cdot{}0^4-2\cdot{}0^2+6\cdot{}^2+4$$
$$f(0)-f(0)=f(0)+4$$
$$0=f(0)+4$$
$$f(0)=-4$$

Отлично, я уже знаю что {\boldmath\color{red} $f(0)=-4$}.

\begin{samepage}
Начну составлять табличку:
\begin{center}
 \begin{tabular}{r r l} 
 \hline
 x & y & Что получилось \\
 \hline
 0 & 0 & $f(0)=-4$      \\
 \hline
 \end{tabular}
\end{center}
\end{samepage}

\subsection{Случай $x=0$, $y=1$}
Продолжу. Теперь попробую {\boldmath $x=0$,$y=1$}:

$$f(x\cdot{}y-2\cdot{}y)-f(2\cdot{}y^2-x)=f(y)+2\cdot{}x^2\cdot{}y^2-8\cdot{}y^4-2\cdot{}x^2+6\cdot{}y^2+4$$


$$f(xy-2y)-f(2y^2-x)=f(y)+2x^2y^2-8y^4-2x^2+6y^2+4$$
$$f(0\cdot{}1-2\cdot{}1)-f(2\cdot{}1^2-0)=f(1)+2\cdot{}0^2\cdot{}1^2-8\cdot{}1^4-2\cdot{}0^2+6\cdot{}1^2+4$$
$$f(-2)-f(2)=f(1)-8+6+4$$
$$f(-2)-f(2)=f(1)-8+10$$
$$f(-2)-f(2)=f(1)+2$$

\begin{samepage}
Пока что я не могу это упростить дальше, так что просто добавлю пока в мою табличку как есть {\boldmath\color{red} $f(-2)-f(2)=f(1)+2$}: 
\begin{center}
 \begin{tabular}{r r l} 
 \hline
 x & y & Что получилось      \\
 \hline
 0 & 0 & $f(0)=-4$           \\
 0 & 1 & $f(-2)-f(2)=f(1)+2$ \\
 \hline
 \end{tabular}
\end{center}
\end{samepage}

\subsection{Случай $x=0$, $y=-1$}
Продолжу. Теперь попробую {\boldmath $x=0$,$y=-1$}:

$$f(x\cdot{}y-2\cdot{}y)-f(2\cdot{}y^2-x)=f(y)+2\cdot{}x^2\cdot{}y^2-8\cdot{}y^4-2\cdot{}x^2+6\cdot{}y^2+4$$
$$f(0\cdot{}-1-2\cdot{}-1)-f(2\cdot{}{-1}^2-0)=f(-1)+2\cdot{}0^2\cdot{}{-1}^2-8\cdot{}{-1}^4-2\cdot{}0^2+6\cdot{}{-1}^2+4$$
$$f(2)-f(2)=f(-1)-8+6+4$$
$$0=f(-1)-8+10$$
$$0=f(-1)+2$$
$$f(-1)=-2$$

\begin{samepage}
Добавляю в табличку новую находку {\boldmath\color{red} $f(-1)=-2$}:
\begin{center}
 \begin{tabular}{r r l} 
 \hline
 x & y & Что получилось       \\
 \hline
 0 &  0 & $f(0)=-4$           \\
 0 &  1 & $f(-2)-f(2)=f(1)+2$ \\
 0 & -1 & $f(-1)=-2$ \\
 \hline
 \end{tabular}
\end{center}
\end{samepage}

\subsection{Случай $x=1$, $y=0$}
$$f(x\cdot{}y-2\cdot{}y)-f(2\cdot{}y^2-x)=f(y)+2\cdot{}x^2\cdot{}y^2-8\cdot{}y^4-2\cdot{}x^2+6\cdot{}y^2+4$$
$$f(1\cdot{}0-2\cdot{}0)-f(2\cdot{}0^2-1)=f(0)+2\cdot{}1^2\cdot{}0^2-8\cdot{}0^4-2\cdot{}1^2+6\cdot{}0^2+4$$
$$f(0)-f(-1)=f(0)-2+4$$
$$f(0)-f(-1)-f(0)=-2+4$$
$$-f(-1)=2$$
$$f(-1)=-2$$
\begin{samepage}
Получилось то же самое, так что в табличку ничего нового:
\begin{center}
 \begin{tabular}{r r l} 
 \hline
 x & y & Что получилось       \\
 \hline
 0 &  0 & $f(0)=-4$           \\
 0 &  1 & $f(-2)-f(2)=f(1)+2$ \\
 0 & -1 & $f(-1)=-2$          \\
 1 &  0 & $f(-1)=-2$ \\
 \hline
 \end{tabular}
\end{center}
\end{samepage}

\subsection{Случай $x=-1$, $y=0$}
$$f(x\cdot{}y-2\cdot{}y)-f(2\cdot{}y^2-x)=f(y)+2\cdot{}x^2\cdot{}y^2-8\cdot{}y^4-2\cdot{}x^2+6\cdot{}y^2+4$$
$$f(-1\cdot{}0-2\cdot{}0)-f(2\cdot{}0^2-(-1))=f(0)+2\cdot{}{-1}^2\cdot{}0^2-8\cdot{}{0}^4-2\cdot{}{-1}^2+6\cdot{}0^2+4$$
$$f(0)-f(1)=f(0)-2+4$$
$$f(0)-f(1)-f(0)=2$$
$$-f(1)=2$$
$$f(1)=-2$$

Отлично! Получается что наша функция в точке {\boldmath\color{red} $f(1)=f(-1)=-2$}. Добавлю это в табличку:

\begin{samepage}
\begin{center}
 \begin{tabular}{r r l} 
 \hline
 x & y & Что получилось        \\
 \hline
  0 &  0 & $f(0)=-4$           \\
  0 &  1 & $f(-2)-f(2)=f(1)+2$ \\
  0 & -1 & $f(-1)=-2$          \\
  1 &  0 & $f(-1)=-2$          \\
 -1 &  0 & $f(1)=-2$ \\
 \hline
 \end{tabular}
\end{center}
\end{samepage}

\subsection{Снова возвращаюсь к случаю $x=0$, $y=1$}
Замечаю, что теперь я могу упростить этот случай, потому что знаю, что $f(1)=-2$:
$$f(-2)-f(2)=f(1)+2, f(1)=-2$$
Следовательно
$$f(-2)-f(2)=-2+2$$
$$f(-2)-f(2)=0$$
$$f(-2)=f(2)$$

\begin{samepage}
Добавлю этот факт в табличку {\boldmath\color{red}$f(-2)=f(2)$} -- обновлю строчку $x=0$, $y=1$:
\begin{center}
 \begin{tabular}{r r l} 
 \hline
 x & y & Что получилось        \\
 \hline
  0 &  0 & $f(0)=-4$           \\
  0 &  1 & $f(-2)=f(2)$        \\
  0 & -1 & $f(-1)=-2$          \\
  1 &  0 & $f(-1)=-2$          \\
 -1 &  0 & $f(1)=-2$ \\
 \hline
 \end{tabular}
\end{center}
\end{samepage}

\subsection{Случай $x=1$, $y=1$}
Иду дальше -- $x=1$, $y=1$:
$$f(x\cdot{}y-2\cdot{}y)-f(2\cdot{}y^2-x)=f(y)+2\cdot{}x^2\cdot{}y^2-8\cdot{}y^4-2\cdot{}x^2+6\cdot{}y^2+4$$
$$f(1\cdot{}1-2\cdot{}1)-f(2\cdot{}1^2-1)=f(1)+2\cdot{}1^2\cdot{}1^2-8\cdot{}1^4-2\cdot{}1^2+6\cdot{}1^2+4$$
$$f(1-2)-f(2-1)=f(1)+2-8-2+6+4$$
$$f(-1)-f(1)=f(1)+2$$
$$f(-1)-(-2)=-2+2$$
$$f(-1)+2=0$$
$$f(-1)=-2$$

\begin{samepage}
Это мы уже видели. Добавлю в табличку:
\begin{center}
 \begin{tabular}{r r l} 
 \hline
 x & y & Что получилось        \\
 \hline
  0 &  0 & $f(0)=-4$           \\
  0 &  1 & $f(-2)=f(2)$        \\
  0 & -1 & $f(-1)=-2$          \\
  1 &  0 & $f(-1)=-2$          \\
 -1 &  0 & $f(1)=-2$           \\
  1 &  1 & $f(-1)=-2$ \\
  \hline
 \end{tabular}
\end{center}
\end{samepage}

\subsection{Случай $x$ как есть, а $y=0$}
Мне надоело тупо перебирать числа. Попробую оставить переменную $x$, а $y$ приравнять к нулю (просто потому что у $y$ есть четвёртая степень а в неё возводить не очень хочется):
$$f(x\cdot{}y-2\cdot{}y)-f(2\cdot{}y^2-x)=f(y)+2\cdot{}x^2\cdot{}y^2-8\cdot{}y^4-2\cdot{}x^2+6\cdot{}y^2+4$$
$$f(x\cdot{}0-2\cdot{}0)-f(2\cdot{}0^2-x)=f(0)+2\cdot{}x^2\cdot{}0^2-8\cdot{}0^4-2\cdot{}x^2+6\cdot{}0^2+4$$
$$f(0)-f(-x)=f(0)-2\cdot{}x^2+4$$
$$f(0)-f(-x)-f(0)=-2\cdot{}x^2+4$$
$$-f(-x)=-2\cdot{}x^2+4$$
$$f(-x)=2\cdot{}x^2-4$$

\begin{samepage}
Отлично! Получилось что-то полезное {\boldmath\color{red}$f(-x)=2\cdot{}x^2-4$}.
\begin{center}
 \begin{tabular}{r r l} 
 \hline
 x & y & Что получилось        \\
 \hline
  0 &  0 & $f(0)=-4$           \\
  0 &  1 & $f(-2)=f(2)$        \\
  0 & -1 & $f(-1)=-2$          \\
  1 &  0 & $f(-1)=-2$          \\
 -1 &  0 & $f(1)=-2$           \\
  1 &  1 & $f(-1)=-2$          \\
  x &  0 & $f(-x)=2\cdot{}x^2-4$ \\
  \hline
 \end{tabular}
\end{center}
\end{samepage}

\subsection{Подбираю $f(14)$}
Из последнего результата можно посчтиать $f(-14)$ но нам нужно $f(14)$, равенство $f(-x)=f(x)$ верно только для чётных функций. Ещё возможен случай нечётной функции, тогда будет что $f(-14)=-f(14)$. Третий случай - функция общего вида, там вообще никакой связи нет. Какой именно случай у нас здесь пока не ясно. Это означает что вообще-то $f(-14) \ne{} f(14)$. Так что напрямую последнюю строчку для вычисления $f(14)$ использовать нельзя.

Можно было-бы попытаться каким-то образом доказать что эта функция чётная, но совершенно неясно как это сделать. С другой стороны -- в табличке уже есть намёки $f(1)=f(-1)=-2$ и $f(-2)=f(2)$. Можно попробовать подобрать такое же для $f(14)$.

Для этого мне нужно найти такие $x$ и $y$ (не просто буквы а уже числа) чтобы одно из выражений в аргументе функции было равно $14$: $f(x\cdot{}y-2\cdot{}y=14)$ или $f(2\cdot{}y^2-x=14)$ или $f(y=14)$.

$f(y=14)$ слишком сложный и в то же время тривиальный случай. Он сложный потому что придётся возводить $14$ в чётвёртую степень, тривиальный потому что другие выражения так и продолжат содержать $x$. Поэтому я этот случай попробую только если ничто другое не сработает.

$f(2\cdot{}y^2-x=14)$ тоже содержит $y^2$ что я вычислять не хочу. 

Таким образом в первую очередь попробую вот что:
$$x\cdot{}y-2\cdot{}y=14$$
$$y(x-2)=14$$

Возьму какие-нибудь конкретные числа. Чтобы избежать возведения больших чисел в четвёртую степень возьму $y=1$, тогда:
$$1\cdot{}(x-2)=14$$
$$x-2=14$$
$$x=16$$

Подставлю найденную пару $x=16$, $y=1$ в исходное выражение:

$$f(x\cdot{}y-2\cdot{}y)-f(2\cdot{}y^2-x)=f(y)+2\cdot{}x^2\cdot{}y^2-8\cdot{}y^4-2\cdot{}x^2+6\cdot{}y^2+4$$
$$f(16\cdot{}1-2\cdot{}1)-f(2\cdot{}1^2-16)=f(1)+2\cdot{}16^2\cdot{}1^2-8\cdot{}1^4-2\cdot{}16^2+6\cdot{}1^2+4$$
$$f(16-2)-f(2-16)=f(1)+2\cdot{}16^2-8-2\cdot{}16^2+6+4$$
$f(1)=-2$, так что:
$$f(14)-f(-14)=-2+2\cdot{}16^2-8-2\cdot{}16^2+10$$
$$f(14)-f(-14)=-2-8+10$$
$$f(14)-f(-14)=-10+10$$
$$f(14)-f(-14)=0$$
$$f(14)=f(-14)$$
Отлично! Это как раз то что нужно было.
\begin{samepage}
Финальный вариант таблички
\begin{center}
 \begin{tabular}{r r l} 
 \hline
 x & y & Что получилось        \\
 \hline
  0 &  0 & $f(0)=-4$           \\
  0 &  1 & $f(-2)=f(2)$        \\
  0 & -1 & $f(-1)=-2$          \\
  1 &  0 & $f(-1)=-2$          \\
 -1 &  0 & $f(1)=-2$           \\
  1 &  1 & $f(-1)=-2$          \\
  x &  0 & $f(-x)=2\cdot{}x^2-4$ \\
 16 &  1 & $f(14)=f(-14)$ \\
 \hline
 \end{tabular}
\end{center}
\end{samepage}

\subsection{Вычисление ответа}
Теперь, используя две последние строчки таблицы, можно просто посчитать:
$$f(14)=f(-14)=2\cdot{}{+14}^2-4=2\cdot{}196-4=392-4=388$$

\section{Ответ}
$$f(14)=388$$

\section{Проверка}
Я не знаю как тут проверить ответ.

\section{Дополнение: сводка методов}
Ничего экстраординарного я тут не использовал. Сначала я подставлял мелкие $x$ и $y$, а потом стал играть с изоляцией одной переменной - $x$, а в конце подобрал такие $x$ и $y$ чтобы они в одном из аргументов давали $14$, значение от которого мы ищем, и это позволило доказать чётность функции в данной точке. При этом всюду я старался избегать возводить большие числа в четвёртую степень.

К сожалению пришлось сделать много бесполезной работы -- для решения достаточно было бы найти всего три случая $f(1)=-2$, $f(-x)=2x^2-4$, $f(14)=f(-14)$, но кто-ж знал?!

% https://habr.com/ru/sandbox/121719/
% http://urok.1sept.ru/%D1%81%D1%82%D0%B0%D1%82%D1%8C%D0%B8/211057/
% https://blog.tutoronline.ru/funkcionalnye-uravnenija
\end{document}

