\documentclass{article}
\usepackage[T1,T2A]{fontenc}
\usepackage[utf8]{inputenc}
\usepackage[english,russian]{babel}
\usepackage{xcolor}
\usepackage{amsmath}
\usepackage{amsfonts,bm}
\usepackage{tikz}
\usetikzlibrary{calc}
\usetikzlibrary{arrows.meta}
\usetikzlibrary{patterns}
\usetikzlibrary{quotes,angles}
\usepackage{listings}
\usepackage{relsize}
\usepackage{float}
\usepackage{wasysym}
\usepackage{MnSymbol}

\title{Решение Задачи №42}
\author{Тани Фуканчик}
\date{Сентябрь 2020}

\begin{document}

\maketitle

\section{Задача}
Решите комплексное уравнение:
$$x^4+1=0$$

\section{Решение}
выделим полный квадрат:
$$(x^2+1)^2=x^4+2x^2+1=x^4+1+2x^2$$
откуда $x^4+1=(x^2+1)^2-2x^2$

Запишу:
$$(x^2+1)^2-2x^2=0$$
по формуле разности квадратов это:
$$(x^2+1-\sqrt{2}x)(x^2+1+\sqrt{2}x)=0$$
откуда
$$x^2+1-\sqrt{2}x=0$$
$$x^2+1+\sqrt{2}x=0$$

В обоих дискриминант одинаковый: $D=2-4=-2$.

из первого уравнения:
$x_{1,2}=\frac{ \sqrt{2}\pm{}\sqrt{-2} } {2}$

заметим, что $\sqrt{-2}=\sqrt{2\cdot{}-1}=\sqrt{2}\sqrt{-1}=\sqrt{2}\cdot{}i$ поэтому

$$x_{1,2}=\frac{ \sqrt{2}(1\pm{}i) } {2}$$
$$x_{1,2}=\frac{ 1\pm{}i } {\sqrt{2}}$$

из второго уравнения:
$$x_{3,4}=\frac{ -\sqrt{2}\pm{}\sqrt{-2} } {2}$$

получаю:
$$x_{3,4}=\frac{ -1\pm{}i } {\sqrt{2} }$$

\section{Ответ}
$$x_{1,2}=\frac{ 1\pm{}i } {\sqrt{2}}$$
$$x_{3,4}=\frac{ -1\pm{}i } {\sqrt{2} }$$

Нарисую все ответы на комплексной плоскости:

\begin{tikzpicture}[>=Straight Barb]
\draw[->,gray] (-2,0) -- (2,0);
\draw[->,gray] (0,-2) -- (0,2);
\node[right] at (0,2) {Im};
\node[below] at (2,0) {Re};

\coordinate(O)at (0,0);
\coordinate[label=right:$\frac{ 1+i } {\sqrt{2}}$](z1)at (1,1);
\coordinate[label=below right:$\frac{ 1-i } {\sqrt{2}}$](z2)at (1,-1);
\coordinate[label=left:$\frac{ -1+i } {\sqrt{2} }$] (z3) at (-1,1);
\coordinate[label=below left:$\frac{ -1-i } {\sqrt{2} }$] (z4) at (-1,-1);
\draw (O)--(z1);
\draw (O)--(z2);
\draw (O)--(z3);
\draw (O)--(z4);
\node at (z1)[orange,circle,fill,inner sep=1.1pt]{};
\node at (z2)[orange,circle,fill,inner sep=1.1pt]{};
\node at (z3)[orange,circle,fill,inner sep=1.1pt]{};
\node at (z4)[orange,circle,fill,inner sep=1.1pt]{};

\draw[gray] (O) circle ({sqrt(2)});
\end{tikzpicture}


\section{Проверка}
Подставлю
$$x_{1}=\frac{ 1+i } {\sqrt{2}}$$

$$\left(\frac{ 1+i } {\sqrt{2}}\right)^4+1=0$$
$$\frac{ (1+i)^4 } {4}+1=0$$
$$(1+i)^4 + 4=0$$
раскрою четвёртую степень по биному Ньютона:
$$1^4{}i^0 + 4\cdot{}1^3{}i^1 + 6\cdot{}1^2{}i^2 + 4\cdot{}1^1{}i^3 + 1^0{}i^4 + 4=0$$
$$1+4\cdot{}i+6\cdot{}i^2+4\cdot{}i^3+i^4+4=0$$
$$1+4i-6-4i+1+4=0$$
$$1-6+1+4=0$$
$$-4+4=0$$
$$0=0$$

Отлично! Проверка сошлась. Аналогично проверяю остальные решения (сокращённо):
$$x_{2}=\frac{ 1-i } {\sqrt{2}}$$
$$\left(\frac{ 1-i } {\sqrt{2}}\right)^4+1=0$$
$$(1-i)^4+4=0$$

$$1^4\cdot{}-i^0 + 4\cdot{}1^3\cdot{}-i^1 + 6\cdot{}1^2\cdot{}(-i)^2 + 4\cdot{}1^1\cdot{}(-i)^3+1^0\cdot{}(-i)^4+4=0$$

$$1-4\cdot{}i-6+4\cdot{}i+1+4=0$$

$$1-6+1+4=0$$
$$-4+4=0$$
$$0=0$$

И это тоже ок.

Иду дальше:
$$x_{3}=\frac{ -1+i } {\sqrt{2} }$$
$$\left(\frac{ -1+i } {\sqrt{2}}\right)^4+1=0$$
$$(-1+i)^4+4=0$$
$$(-1)^4\cdot{}i^0 + 4\cdot{}(-1)^3\cdot{}i^1 +6\cdot{}(-1)^2\cdot{}i^2 + 4\cdot{}(-1)^1\cdot{}i^3 + (-1)^0\cdot{}i^4 + 4=0$$

$$1-4{}i-6+4{}i+1+4=0$$
$$1-6+1+4=0$$
$$-4+4=0$$
$$0=0$$

И это ок.

И последнее:
$$x_{3}=\frac{ -1-i } {\sqrt{2} }$$
$$\left(\frac{ -1-i } {\sqrt{2}}\right)^4+1=0$$
$$(-1-i)^4+4=0$$
$$(-1)^4\cdot{}(-i)^0 + 4\cdot{}(-1)^3\cdot{}(-i)^1 + 6\cdot{}(-1)^2\cdot{}(-i)^2 + 4\cdot{}(-1)^1\cdot{}(-i)^3+ (-1)^0\cdot{}(-i)^4 + 4 = 0$$

$$1+4i-6-4i+1+4=0$$
$$1-6+1+4=0$$
$$-4+4=0$$
$$0=0$$
отлично! Все 4 ответа сошлись.

\end{document}

