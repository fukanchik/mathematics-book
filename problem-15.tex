\documentclass{article}
\usepackage[T1,T2A]{fontenc}
\usepackage[utf8]{inputenc}
\usepackage[english,russian]{babel}
\usepackage{xcolor}
\usepackage{amsmath}
\usepackage{amsfonts}
\usepackage{tikz}
\usetikzlibrary{patterns}
\usepackage{relsize}

\title{Решение Задачи №15}
\author{Серёжи Фуканчика\\34 Ю класс}
\date{Декабрь 2019}

\begin{document}

\maketitle

\section{Задача}
С лодки, движущейся со скоростью 2 м/с человек бросает весло массой 5 кг с горизонтальной скоростью 8 м/с противоположно движению лодки. С какой скоростью стала двигаться лодка после броска если её масса вместе с человеком равна 200 кг?

\section{Решение}
Это очевидно задача на импульс. До броска был один импульс, после броска - другой. До броска было единое тело -- лодка, в ней человек и весло. Они двигались в одном направлении с одинаковой скоростью.

Считаю, что горизонтальная скорость брошенного весла дана относительно берега а не относительно лодки.

Задача одномерного движения. Удобно рассматривать систему отсчёта связанную с берегом. Лодка движется в положительном направлении.

Рассмотрим ситуацию ДО броска. Суммарная масса лодки, человека и весла - 205 кг. Всё это движется со скоростью 2 м/с в положительном направлении. Импульс равен:
$$P_{\textrm{исх.}}=205\cdot{}2=410$$

ПОСЛЕ броска система разделилась на две части: лодка и человек массой 200 кг и некоторой скоростью $v$ и весло массой 5 кг и скоростью -8 м/с. Минус здесь потому что весло летит в противоположную сторону. Полный импульс выглядит так:
$$P_{\textrm{после}}=200\cdot{}v+5\cdot-8$$
Закон сохранения импулся позволяет приравнять импульсы ДО и ПОСЛЕ. Приравниваю:
$$410=200\cdot{}v+5\cdot-8$$
$$200\cdot{}v=450$$
$$v=4.5/2=2.25 \textrm{ м/с}$$
\section{Ответ}
$$v=2.25 \textrm{ м/с}$$

\end{document}

